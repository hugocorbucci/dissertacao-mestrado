%% ------------------------------------------------------------------------- %%
\chapter{Métodos Ágeis Abertos para o OMM}
\label{cap:omm}

O objetivo do projeto QualiPSo\footnote{\url{http://www.qualipso.org}
  - Último acesso em 27/08/2010} é de aumentar a confiabilidade da
indústria e a qualidade dos sistemas livres existentes e futuros. Para
atingir esse objetivo, o projeto conta com 10 grandes áreas de
trabalho. Uma dessas áreas é relacionado à confiabilidade do processo
usado no desenvolvimento de projetos livres.

Desde o início, o projeto abraçou o fato de que não poderia jamais
forçar uma forma de trabalho a comunidades livres. Por isso, a
abordagem usada para aumentar essa confiabilidade foi estabelecer uma
forma de avaliar a qualidade do processo usado por um determinado
projeto livre. Sendo assim, o projeto procurou elaborar um selo que
pudesse ser dado às comunidades que estivesse de acordo com um modelo
de processo confiável.

Porém, o contexto de projetos livre difere (como foi apresentado
anteriormente) do contexto para ambientes empresariais comuns. Por
isso, modelos de avaliação de processos estabelecidos na indústria não
são adequados para ambientes livres. Por isso, decidiu-se elaborar o
modelo de maturidade para software livre do Qualipso (\textit{QualiPSo
  Opensource Maturity Model} - OMM).

A seção \ref{sec:o-que-eh-omm} apresenta mais detalhes da origem do
OMM e de sua constituição. Em seguida, a seção \ref{sec:xp-em-omm}
apresenta como programação extrema pode ser mapeada para o OMM e quais
são os pontos não tratados. Por fim, a seção
\ref{sec:openagile-em-omm} apresenta uma sugestão de práticas
complementares para permitir a aprovação no OMM.

\section{Origem e descrição do OMM}
\label{sec:o-que-eh-omm}

O OMM se baseia na ideia de que a indústria confia em certificados de
qualidade. Padrões como o selo
ISO9001\footnote{\url{http://www.iso.org/} - Último acesso em
  27/08/2010} ou como o Modelo de Maturidade de Capabilidade
(\textit{Capability Maturity Model} - CMM) do Instituto de Engenharia
de Software ( \textit{Software Engineering Institute} - SEI)
\footnote{\url{http://www.sei.cmu.edu/cmmi} - Último acesso em
  27/08/2010} são constituídos de documentos que descrevem uma lista
de exigências que precisam ser cumpridas pelo processo das empresas
que esperaraem obter o selo.

Como projetos livres raramente beneficiam de uma infraestrutura física
ou organizacional, é muito difícil avaliar esses processos de acordo
com esses padrões da indústria. Por isso, o QualiPSo propôs trabalhar
num modelo baseado no CMM mas que pudesse ser usado não apenas para
empresas que incluem software livre em suas soluções mas também pelas
comunidades livres ao redor do mundo. Desse fato, decorre uma nota
importante sobre o OMM. O modelo todo foi pensado para que fosse
simples e fácil de usar pelos vários níveis organizacionais existentes
no ambiente de softwre livre.

A primeira fase de elaboração do OMM foi realizar um levantamento dos
chamados elementos de confiabilidade (\textit{Trusthworthy elements})
no contexto de software livre. Os elementos identificados formaram a
base do OMM para garantir que o processo avaliado não apresentasse
apenas qualidade e confiabilidade do ponto de vista comercial mas
também no contexto de comunidades livres.

Levantados esses elementos de confiabilidade, a equipe do OMM realizou
um mapeamento das áreas de qualidades avaliadas no CMM para
identificar quais elementos eram abordados e quais não eram. Os
principais elementos de confiabilidade que o CMM não aborda estão
relacionados aos problemas legais do uso de software e à reputação de
determinado projeto além do tamanho de sua comunidade.

No aspecto legal, as questões do licenciamento do código, da violação
de patentes e preservação de marcas são pontos importantíssimos para
permitir o uso de qualquer projeto livre numa organização
comercial. Na questão das contribuições, é importante tomar cuidado
com a questão dos direitos autorais para evitar problemas legais
relacionados ao licenciamento do código.  Esses dois aspectos não são
tratados ou sequer abordados no CMM porque a existência de uma
organização responsável por qualquer desenvolvimento e de regras
contratuais estabelecidas evita esses problemas.

Por outro lado, o CMM aborda alguns aspectos que são importantes para
a confiabilidade de um projeto no contexto comercial. Muitos desses
aspectos estão ligados a exigências na quantidade e detalhamento de
documentos usados para inspeção e melhoria dentro da organização que
implementa o processo. A equipe do OMM selecionou todas as práticas
sugeridas pelo CMM no que diz respeito aos aspectos técnicos e apenas
algumas no aspecto gerencial que fizessem sentido no contexto livre.

Graças a esse trabalho, o OMM foi formado com um misto de elementos de
confiabilidade vindos da comunidade de software livre com práticas
estabelecidas vindas do CMM. O modelo ainda optou por adotar uma
estrutura piramidal semelhante à do CMM na qual existem três níveis de
adequação sendo que o mais básico é base para os mais avançados que
sempre exigem todas as práticas do nível inferior e mais algumas.

A figura \ref{fig:piramide-omm} apresenta a divisão de níveis com a
lista de práticas (com nomes abreviados) de integra cada um dos níveis
do OMM. Além disso, o OMM propõe exigências diferentes de acordo com o
tipo de entidade que deseja ser avaliada para um determinado
nível. Isto é, algumas práticas são apenas recomendadas e não
obrigatórias para comunidades livres não representadas por uma
empresa. Dessa forma, quando o projeto não tem uma organização por
trás, os membros da comunidade só precisam realizar o que está no
alcance de uma comunidade para atingir um determinado nível.

% TODO Incluir a imagem

As tabelas \ref{tab:omm-basic}, \ref{tab:omm-intermediate} e
\ref{tab:omm-advanced} mostram os elementos de confiabilidade que
precisam ser endereçados para se atingir os níveis básico,
intermediário e avançados do OMM.

% TODO Incluir as tabelas

\section{Um mapeamento de Programação Extrema para o OMM}
\label{sec:xp-em-omm}


\section{Um proposta ágil compatível com o OMM}
\label{sec:openagile-em-omm}

