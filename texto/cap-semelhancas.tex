%% ------------------------------------------------------------------------- %%
\chapter{Semelhanças entre Métodos Ágeis e Software Livre}
\label{cap:semelhancas}

Métodos ágeis e software livre tem formas de trabalho tão semelhantes
que software livre já foi considerado um método ágil por Martin Fowler
na sua primeira versão de \emph{``The New Methodology''}
\cite{Fowler00orig}. No entanto, Fowler retirou as comunidades de
software livre de seu artigo para a publicação pois considerou que
faltava uma descrição mais precisa dos métodos de desenvolvimento
usados por essas comunidades. Mais tarde, Warsta et
al. \cite{Warsta2002} apresentou um relatório técnico sobre
metodologias de desenvolvimento ágil nas quais incluiu software
livre. A inclusão ainda levou à elaboração de um artigo
\cite{Warsta2003} em que os autores apontam fortes semelhanças entre
métodos ágeis e software livre e concluem que desenvolvimento de
projetos livres pode ser enxergado como uma das facetas associadas aos
métodos ágeis.

Até hoje, a principal referência para descrever métodos de
desenvolvimento de projetos livres é a de Eric Raymond em \emph{``The
  Cathedral and the Bazaar''} \cite{Raymond1999}. O texto traz o
relato de algumas experiências vividas por Raymond e as decisões que
levaram seus projetos livres ao sucesso. Várias dessas decisões e as
ideias por trás delas podem ser relacionadas ao manifesto ágil
\cite{AgileManifesto}.

O manifesto é constituido de um texto principal que destaca quatro
valores e de uma lista de doze princípios que apoiam esses valores. O
texto principal é curto e muito conhecido e pode ser conferido na
caixa \ref{box:manifesto}.

\begin{caixa}[hb]
  \begin{minipage}{\linewidth}
    \centering Estamos descobrindo maneiras melhores de desenvolver
    software, fazendo-o nós mesmos e ajudando outros a fazerem o
    mesmo. Através deste trabalho, passamos a valorizar:

    \begin{center}
      \begin{itemize}
      \item \textbf{Indivíduos e interações} mais que processos e
        ferramentas;
      \item \textbf{Software em funcionamento} mais que documentação
        abrangente;
      \item \textbf{Colaboração com o cliente} mais que negociação de
        contratos e
      \item \textbf{Responder a mudanças} mais que seguir um plano.
      \end{itemize}
    \end{center}

    Ou seja, mesmo havendo valor nos itens à direita, valorizamos mais
    os itens à esquerda.
  \end{minipage}
  \caption{Manifesto ágil}
  \label{box:manifesto}
\end{caixa}

As próximas quatro seções apresentam a relação entre as atitudes
encontradas na maioria das comunidades de software livre e cada um dos
quatro valores enunciados pelo manifesto apoiando-se nas frases
apresentadas. A Seção \ref{sec:agile?} apresenta o resumo dos
argumentos e descreve os pontos onde podem existir algumas falhas.

\section{Indivíduos e interações são mais importantes que processos e
  ferramentas}
\label{sec:first-princ}

Várias pesquisas relacionadas a desenvolvimento de software livre
apresentam uma quantidade razoável de ferramentas usadas por
desenvolvedores para manter a comunicação entre os membros da
equipe. Reis \cite{Reis2003} mostra que 65\% dos projetos analisados
usam programas de controle de versão, a página na Internet do projeto
e listas de correio eletrônico como as principais ferramentas de
comunicação entre os usuários do programa e a equipe de
desenvolvimento. \textbf{Os processos e ferramentas} são, no entanto,
apenas um meio de atingir um objetivo: garantir um ambiente estável e
acolhedor para a criação do programa de forma colaborativa.

Apesar dos negócios baseados em software livre estarem crescendo, a
essência da comunidade ao redor do programa é de manter
\textbf{indivíduos que interajam} de forma a produzir o que lhes
interessa. As ferramentas apenas possibilitam isso. Nessas
comunidades, interações normalmente ocorrem para que os indivíduos
contribuam com código fonte e com documentação, independente do modelo
de negócios. Essas atividades são responsáveis por dirigir o processo
e modificar as ferramentas para que elas cumpram melhor as
necessidades da comunidade.

\section{Software em funcionamento é mais importante que documentação
  abrangente}
\label{sec:second-princ}

De acordo com Reis \cite{Reis2003}, 55\% dos projetos de software
livre atualizam ou revisam suas documentações frequentemente e 30\%
mantêm documentos que explicam como certas partes do sistema funcionam
ou como o projeto está organizado. Esses resultados mostram que a
documentação para os usuários é considerada importante mas não é o
objetivo final dos projetos. Por outro lado, é muito comum encontrar
projetos de software livre onde os requisitos do sistema são descritos
como \emph{bugs} no sentido de que representam alguma coisa no
software que não funciona da forma que deveria.

Mais recentemente, Oram \cite{Oram2007} apresentou os resultados de
uma pesquisa organizada pela O'Reilly mostrando que documentação de
software livre está, cada vez mais, sendo escrita por
voluntários. Isso significa que \textbf{documentação completa e
  detalhada} cresce com a comunidade ao redor de \textbf{software
  funcionando}, conforme os usuários encontram problemas para
completar determinada ação. De acordo com o trabalho de Oram, os
principais motivos para que contribuidores escrevam documentação é
para seu crescimento pessoal ou para melhorar o nível da
comunidade. Essa motivação explica porque a documentação de software
livre normalmente abrange muito bem os problemas mais comuns e explica
como usar as principais funcionalidades mas deixam a desejar quanto se
trata de problemas ou funcionalidades menos comuns ou usados.

\section{Colaboração com o cliente é mais importante que negociação de
  contratos}
\label{sec:third-princ}

\textbf{Negociação de contratos} ainda é um problema apenas para uma
quantidade muito pequena de projetos de software livre já que a grande
maioria deles não envolve nenhum contrato.  Por outro lado, o modelo
proposto pelo
SourceForge.net\footnote{\url{http://www.sourceforge.net/} - Último
  acesso 28/10/2010} é de contratação de um ou mais desenvolvedores
para o desenvolvimento de uma determinada funcionalidade por um curto
período de tempo. Neste contrato, o desenvolvedor presta um serviço ao
cliente desenvolvendo a funcionalidade e integrando ela ao
projeto. Apesar do modelo de negócio não garantir que o cliente irá
participar ativamente e colaborar com a equipe, o seu curto prazo faz
com que o intervalo de tempo entre as conversas seja pequeno,
aumentando, por tanto, o \emph{feedback} e reduzindo a força de um
contrato mais rígido.

O ponto chave nessa questão é que a colaboração é a base dos projetos
de software livre.  O cliente se envolve no projeto o quanto ele
desejar. \textbf{Clientes podem colaborar} mas eles não são
especialmente encorajados a fazê-lo ou obrigados a isso. Isso pode ser
relacionado com a pouca experiência que essas comunidades têm com
relacionamento com clientes. No entanto, vários projetos de sucesso
dependem de sua habilidade de prover respostas rápidas às
funcionalidades pedidas pelos usuários. Nesse caso, a colaboração do
usuário (ou do cliente) aliada com a habilidade de responder
rapidamente aos pedidos é especialmente poderosa.

\section{Responder a mudanças é mais importante que seguir um plano}
\label{sec:fourth-princ}

Uma busca no Google por \emph{``Development Roadmap open source''}
respondeu com mais de 943.000 resultados no dia 28/10/2010. Isso
sugere que muitos projetos de software livre costumam publicar seus
planos para o futuro. No entanto, \textbf{seguir estes planos} não é
uma regra. Pior, ater-se demais a esse plano pode levar um projeto a
ser abandonado pelos seus usuários ou colaboradores.

O principal motivo para isso é o ambiente extremamente competitivo do
universo de software livre no qual apenas os melhores projetos
conseguem sobreviver e atrair colaboração. A \textbf{habilidade de
  cada projeto em se adaptar e responder às mudanças} é crucial para
determinar os projetos que sobrevivem.  Projetos que não se adaptam às
mudanças são abandonados pelos seus usuários em prol de outros
projetos mais atualizados e que atendem melhor às suas necessidades.
Um dos maiores exemplos deste fato foi a queda do navegador Netscape
quando, pressionado pelo Internet Explorer, a empresa deixou de
investir em desenvolvimento e perdeu a maior parte da sua base de
usuários. Anos depois, o Firefox emergiu dos restos do Netscape e
conquistou milhões de usuários pelas suas atualizações frequentes e
funcionalidades inovadoras.

\section{Aproximação de Software Livre com Métodos Ágeis}
\label{sec:agile?}

Apesar dos pontos do manifesto ágil serem seguidos e apoiados em
várias comunidades de software livre, não há nada que possa ser
qualificado como um método de desenvolvimento de software livre. A
descrição de Raymond \cite{Raymond1999} é um ótimo exemplo de como o
processo pode funcionar mas ele não descreve práticas ou recomendações
para que outros atinjam o mesmo sucesso. Se uma descrição cuidadosa de
um processo para software livre fosse escrita, ela deveria juntar as
ideias apresentadas por Raymond com uma definição de um processo.

Esse processo obviamente não poderia descrever a forma com que a
maioria dos projetos existentes trabalha mas poderia nortear futuros
projetos. O processo sugerido seguiria as mesmas regras de seleção que
os próprios projetos. Se ele fosse útil para uma certa quantidade de
pessoas, ele seria adotado e difundido por uma comunidade que se
encarregaria de melhorá-lo e corrigi-lo ao longo do tempo. As
ferramentas e suportes necessários para adoção completa do processo
também seriam tomados a cargo da comunidade ao longo do tempo. Caso o
processo não fosse útil o suficiente para os usuários, ele seguiria o
caminho de muitos outros projetos: o esquecimento.

As comunidades criadas ao redor de projetos de software livre envolvem
usuários, desenvolvedores e, algumas vezes, até clientes trabalhando
juntos para talhar o melhor software possível para seus objetivos. A
ausência de tal comunidade ao redor de um programa normalmente é
evidência de que o projeto é recente ou está morrendo. As equipes de
desenvolvimento devem estar muito atentas a esse tipo de sinais que a
comunidade do seu software dá pois eles mostram a saúde do
projeto. Atualmente, preocupações relacionadas a esse aspecto do
desenvolvimento de software livre não são propriamente abordadas pelos
métodos ágeis mais conhecidos.

Com esta análise da relação entre os valores de métodos ágeis
encontrados em software livre em mente, o próximo capítulo (Capítulo
\ref{cap:pesquisas}) apresenta duas pesquisas elaboradas para definir
melhor a relação entre as duas comunidades. Também serão apresentados
os resultados coletados no trabalho e a conclusão à qual eles levaram.
