%% ------------------------------------------------------------------------- %%
\chapter{Conclusões}
\label{cap:conclusoes}

% TODO Mencionar vantagens além do trabalho de Warsta

Existem muitas semelhanças nos valores da comunidades de métodos ágeis
e de software livre. Em ambos os casos, alguns dos fundadores de cada
comunidade dizem ter iniciado seus respectivos movimentos para fugir
do estado da indústria de software em suas épocas. Raymond afirma que
entrou no movimento de software livre para nunca mais precisar se
submeter às pressões da indústria de software que o impediam de
desenvolver código decentemente. Do outro lado, os autores do
manifesto ágil afirmam que escreveram o manifesto para incentivar a
indústria de desenvolvimento a mudar suas formas de trabalho para que
as empresas dessem o tempo aos programadores de fazer seu trabalho.

Essas vontades iniciais semelhantes deixaram rastros importantes na
forma de trabalho existente. O Capítulo~\ref{cap:semelhancas}
apresentou como as comunidades de software livre abraçam os valores
listados no manifesto ágil. Desses valores, destacam-se, em essência,
uma forte confiança nos indivíduos e suas capacidades de resolverem
seus próprios problemas, uma minimização da importância de processos,
ferramentas ou documentos e uma atenção redobrada à qualidade do
produto final do desenvolvimento.

Apesar dessas raízes e valores comuns, os contextos ótimos e soluções
encontradas para os problemas decorrentes desses contextos são BEM
diferentes. Comunidades de software livre são naturalmente
distribuídas, envolvem pouquíssimas relações pessoais face a face,
contam com o poder da falha e redundância e se atacam principalmente a
problemas que a afligem. Agilistas dão preferência para equipes
pequenas situadas no mesmo local, com uso extenso de comunicação
direta e informal e evolução contínua com atenção extrema para
minimizar o trabalho realizado.

Ambas propostas tiveram um certo êxito em seus respectivos nichos e,
com seu sucesso, começaram a explorar contextos diferentes, outros
problemas e até mesmo o simples crescimento dos problemas enfrentados
com eles. Com isso, métodos ágeis rapidamente chegaram a problemas que
exigiam equipes distribuídas e o uso ou evolução de sistemas
desenvolvidos por terceiros. Empresas começaram a procurar integrar e
evoluir projetos livres com equipes próprias e para projetos que
buscam resolver problemas fora da áurea dos problemas de
desenvolvimento tradicionalmente abordados.

Dessa forma, pareceu natural tentar aproximar ambas comunidades e
descobrir o que poderia dificultar essa união. Os resultados das
pesquisas apresentados no Capítulo~\ref{cap:pesquisas} indicaram que
as comunidades, de fato, são constituídas de pessoas diferentes com
costumes e atividades diferentes. No entanto, os problemas
identificados e as ferramentas apropriadas para tratá-los foram
impressionantemente semelhantes.  A dicotomia formada pela semelhança
das soluções e a diferença nos costumes e atividades levou a uma
análise mais aprofundada dos princípios que norteiam os membros de
cada comunidade.

Esses princípios, apresentados no Capítulo~\ref{cap:diferencas},
evidenciam as diferenças de origens e contextos ótimos de cada
comunidade deixando, no entanto, algumas boas práticas comuns
emergirem. Dessas boas práticas, notam-se especialmente as práticas
que valorizam os indivíduos envolvidos e aumentam a confiança
depositada neles. Já nas diferenças, destacam-se os princípios que
busquem, do lado de métodos ágeis, corrigir e adaptar constantemente o
desenvolvimento em busca do objetivo e, do lado de software livre,
multiplicar, diversificar e selecionar as soluções de forma a atingir
o objetivo (ao estilo da seleção natural de Darwin).

Novamente, nas semelhanças, percebe-se uma ligação que permite tirar
proveito das diferenças existentes para beneficiar cada uma das
comunidades.Os agilistas, num contexto que exija distribuição e a
impossibilidade de controlar os membros da equipe de desenvolvimento,
podem aproveitar-se de um novo papel em suas equipes (o
\emph{commiter} - Seção~\ref{subsec:commiter}), de uma política geral
de envolvimento de todos para o projeto (resultados públicos -
Seção~\ref{subsec:publicity}) e de um processo de evolução de
arquitetura que ajude tanto a atingir uma solução melhor quanto
documentar o processo de decisão (revisão cruzada -
Seção~\ref{subsec:crossrev}).  Já do lado dos contribuidores de
software livre, os maiores benefícios vem de ferramentas e atividades
que permitam melhorar a comunicação e o senso de equipe dos
envolvidos. Ambientes (virtuais) que mantenham todos os membros da
equipe atualizados dos avanços no desenvolvimento do projeto (Ambiente
Informativo - Seção~\ref{subsec:inform-worksp}), formas de descrever e
planejar o trabalho a ser realizado de forma mais sucinta para
permitir maiores mudanças (Histórias - Seção~\ref{subsec:stories}),
momentos para para e pensar sobre o trabalho realizado e como ele pode
ser mais proveitoso, prazeroso ou eficiente (Retrospectivas -
Seção~\ref{subsec:retrospect}) e ferramentas que permitam manter
informações importantes do dia-dia das atividades disponíveis para
todos (Papo em Pé - Seção~\ref{subsec:stand-up}).

Munidos dessas práticas adaptadas e das originais descritas pela
Programação Extrema, propõe-se uma união parcial dessas comunidades em
um contexto misto para ambas. Esse contexto foi traçado pelo projeto
QualiPSo com seu objetivo de aumentar a confiança das empresas para
adoção, uso e desenvolvimento de projetos livres.  Nessa situação, o
contexto ótimo de métodos ágeis (empresas buscando evoluir projetos de
risco com sucesso) penetra no mundo do software livre (poucos controle
sobre as contribuições, direções e adoções de um projeto). Nesse
amalgama aparentemente ideal para a união das duas comunidades, o
Capítulo~\ref{cap:omm} apresentou uma proposta de adequação de
Programação Extrema para permitir sua adoção conforme às exigências do
Modelo de Maturidade Aberto desenvolvido no projeto QualiPSo, o OMM.

Dada a natureza piramidal do modelo e suas origens no sistema de
certificação do CMM, atingir os últimos níveis de maturidade
determinados é uma tarefa árdua que requer, de uma forma ou de outra,
um enrigecimento do processo de desenvolvimento que vai depender muito
de cada comunidade envolvida. A proposta apresentada busca apenas
apresentar os argumentos iniciais que permitam justificar a adoção de
um método ágil como uma forma de garantir o nível básico do
modelo. Ela apresentou também como algumas práticas permitem ir além
desse nível e cumprir algumas exigências mais avançadas sem, no
entanto, ser suficiente para cumprir todos os requisitos. Por fim,
ressaltou-se que, pela juventude do modelo de maturidade, sua
descrição e preocupações ainda devem ser modificadas conforme outras
propostas forem escritas, submetidas e aprovadas.

Como regra geral, esse trabalho mostrou que existem muitos pontos em
comum entre o universo dos métodos ágeis e do software livre. No
entanto, a abordagem geral em busca de melhores formas de desenvolver
software e o dia-dia de cada uma das comunidades difere
sensivelmente. Apesar disso, não é possível, ainda, concluir nada a
respeito do futuro dessas comunidades e dos resultados de uma possível
mistura dessas duas comunidades.

A partir dos resultados levantados nesse trabalho, existem alguns
possíveis caminhos para trabalhos futuros. Para cada uma das
ferramentas identificadas, um possível trabalho envolve a
implementação dessa ferramenta, implantação da mesma em incubadoras de
projetos livres e monitoramento da evolução dos projetos que utilizam
e não utilizam a ferramenta. Dessa forma, um estudo experimental
poderia validar (ou invalidar) as propostas apresentadas aqui.  De
forma semelhante, a aplicação das práticas adaptadas descritas para
cada uma das comunidades poderia ter seu valor avaliado graças a um
estudo experimental a seu respeito. As práticas vindas do contexto de
software livre em direção ao de métodos ágeis oferecem provavelmente
uma maior facilidade para serem estudadas já que demandam apenas duas
equipe (uma que adota a prática e uma que não a adota). Já as práticas
sugeridas para ambientes de projetos livres envolvem o monitoramento
de comunidades incontroláveis e cuja observação é muito mais
complicada. Por fim, do lado do OMM e do projeto QualiPSo, o ideal
seria que alguma comunidade de software livre procurasse obter o
certificado de nível básico do OMM utilizando essencialmente algum
processo decorrente do que foi descrito aqui. Essa empreitada exigiria
não apenas uma comunidade ativa e engajada como uma empresa disposta a
iniciar o processo e prover o suporte necessário nos aspectos não
abordados pela Programação Extrema.
