% Arquivo LaTeX de exemplo de dissertação/tese a ser apresentados à CPG
% do IME-USP
% 
% Versão 3: Thu Dec 10 23:57:33 BRST 2009

%
% Criação: Jesús P. Mena-Chalco
% Revisão: Fábio Kon
%  
% Obs: Leia previamente o texto do arquivo README.txt
%

\documentclass[11pt,twoside,a4paper]{book}

% ---------------------------------------------------------------------------- %
% Pacotes 
\usepackage[T1]{fontenc}
\usepackage[brazil]{babel}
\usepackage[utf8]{inputenc}
\usepackage[pdftex]{graphicx}           % usamos arquivos pdf/png como figuras
\usepackage{pifont}
\usepackage{amsfonts}
\usepackage{amssymb} 
\usepackage{setspace}                   % espaçamento flexível
\usepackage[bf,small,compact]{titlesec} % cabeçalhos dos títulos: menores e compactos
\usepackage{indentfirst}                % indentação do primeiro parágrafo
\usepackage{subfigure}                  % uso de várias figuras numa só
\usepackage{makeidx}                    % índice remissivo
\usepackage[nottoc]{tocbibind}          % acrescentamos a bibliografia/indice/conteudo no Table of Contents
\usepackage{courier}                    % usa o Adobe Courier no lugar de Computer Modern Typewriter
\usepackage{type1cm}                    % fontes realmente escaláveis
\usepackage{listings}                   % para formatar código-fonte (ex. em Java)
\usepackage{setspace}
\usepackage{longtable}
\usepackage{multirow}
\usepackage{titletoc}
\usepackage{lscape}
\usepackage[fixlanguage]{babelbib}
\usepackage[font=small,format=plain,labelfont=bf,up,textfont=it,up]{caption}
\usepackage[usenames,svgnames,dvipsnames,table]{xcolor}
\usepackage[a4paper,top=2.54cm,bottom=2.0cm,left=2.0cm,right=2.54cm]{geometry} % margens
\usepackage[pdftex,plainpages=false,pdfpagelabels,pagebackref,colorlinks=true,citecolor=black,linkcolor=black,urlcolor=black,filecolor=black,bookmarksopen=true]{hyperref} % links em preto
%\usepackage[pdftex,plainpages=false,pdfpagelabels,pagebackref,colorlinks=true,citecolor=DarkGreen,linkcolor=NavyBlue,urlcolor=DarkRed,filecolor=green,bookmarksopen=true]{hyperref} % links coloridos
\usepackage[all]{hypcap}                % soluciona o problema com o hyperref e capitulos
\usepackage[numbers,square,sort,nonamebreak,comma]{natbib}  % citação
                                % bibliográfica alpha (alpha-ime.bst)
\usepackage{float}

\usepackage{type1cm}      % fontes realmente escaláveis
\fontsize{60}{62}\usefont{OT1}{cmr}{m}{n}{\selectfont}

% ---------------------------------------------------------------------------- %
\floatstyle{boxed}
\newfloat{caixa}{htb}{bx}
\floatname{caixa}{Caixa}

% ---------------------------------------------------------------------------- %
% headers similares oa TAOP de Donald E. Knuth
\usepackage{fancyhdr}
\pagestyle{fancy}
\fancyhf{}
\renewcommand{\chaptermark}[1]{\markboth{\MakeUppercase{#1}}{}}
\renewcommand{\sectionmark}[1]{\markright{\MakeUppercase{#1}}{}}
\renewcommand{\headrulewidth}{0pt}

% ---------------------------------------------------------------------------- %
\graphicspath{{./figuras/}}             % path das figuras (recomendável)
\frenchspacing                          % Arruma o espaço: id est (i.e.) e exempli gratia (e.g.) 
\urlstyle{same}                         % URL com o mesmo estilo do texto e nao mono-spaced
\makeindex                              % para o índice remissivo
\raggedbottom                           % para não permitir espaços extra no texto
\fontsize{60}{62}\usefont{OT1}{cmr}{m}{n}{\selectfont}
\cleardoublepage
\normalsize

% ---------------------------------------------------------------------------- %
% Opções de listing usados para o código fonte
% Ref: http://en.wikibooks.org/wiki/LaTeX/Packages/Listings
\lstset{ %
language=Java,                  % choose the language of the code
basicstyle=\footnotesize,       % the size of the fonts that are used for the code
numbers=left,                   % where to put the line-numbers
numberstyle=\footnotesize,      % the size of the fonts that are used for the line-numbers
stepnumber=1,                   % the step between two line-numbers. If it's 1 each line will be numbered
numbersep=5pt,                  % how far the line-numbers are from the code
showspaces=false,               % show spaces adding particular underscores
showstringspaces=false,         % underline spaces within strings
showtabs=false,                 % show tabs within strings adding particular underscores
frame=single,	                % adds a frame around the code
framerule=0.6pt,
tabsize=2,	                    % sets default tabsize to 2 spaces
captionpos=b,                   % sets the caption-position to bottom
breaklines=true,                % sets automatic line breaking
breakatwhitespace=false,        % sets if automatic breaks should only happen at whitespace
escapeinside={\%*}{*)},         % if you want to add a comment within your code
backgroundcolor=\color[rgb]{1.0,1.0,1.0}, % choose the background color.
rulecolor=\color[rgb]{0.8,0.8,0.8},
extendedchars=true,
xleftmargin=10pt,
xrightmargin=10pt,
framexleftmargin=10pt,
framexrightmargin=10pt
}

% ---------------------------------------------------------------------------- %
% Corpo do texto
\begin{document}
\frontmatter 
% headers para as páginas do frontmatter 
\fancyhead[RO]{{\footnotesize\rightmark}\hspace{2em}\thepage}
\setcounter{tocdepth}{2}
\fancyhead[LE]{\thepage\hspace{2em}\footnotesize{\leftmark}}
\fancyhead[RE,LO]{}
\fancyhead[RO]{{\footnotesize\rightmark}\hspace{2em}\thepage}

\onehalfspacing  % espaçamento


% ---------------------------------------------------------------------------- %
% Capa
\thispagestyle{empty}
\begin{center}
  \vspace*{2.3cm}
  \textbf{\Large{Métodos ágeis e software livre:\\
      Um estudo do relacionamento entre\\
    estas duas comunidades}}\\
	
  \vspace*{1.2cm} \Large{Hugo Corbucci}
    
  \vskip 2cm \textsc{
    Dissertação apresentada\\[-0.25cm]
    ao\\[-0.25cm]
    Instituto de Matemática e Estatística\\[-0.25cm]
    da\\[-0.25cm]
    Universidade de São Paulo}
    
  \vskip 1.5cm
  Programa: Mestrado em Ciência da Computação\\
  Orientador: Prof. Dr. Alfredo Goldman

  \vskip 1cm \normalsize{Durante o desenvolvimento deste trabalho o
    autor recebeu auxílio financeiro do projeto Qualipso}
	
  \vskip 0.5cm \normalsize{São Paulo, Janeiro de 2010}
\end{center}

% ---------------------------------------------------------------------------- %
% Página de rosto (só para a versão final) \newpage
% \thispagestyle{empty}
%	\begin{center}
%   \vspace*{2.3 cm}
%   \textbf{\Large{Título do trabalho a ser apresentado à \\
%       CPG para a dissertação/tese}}\\
%   \vspace*{2 cm}
%	\end{center}
%
%	\vskip 2cm
%
%	\begin{flushright}
%   Este exemplar corresponde à redação\\
%   final da dissertação devidamente corrigida\\
%   e defendida por Hugo Corbucci\\
%   e aprovada pela Comissão Julgadora.  \vskip 2cm
%
%	\end{flushright}
%	\vskip 4.2cm
%
%	\begin{quote}
%   \noindent Banca Examinadora:
%	
%   \begin{itemize}
%   \item Prof. Dr. Alfredo Goldman (orientador) - IME-USP.
%   \item Prof. Dr. Fabio Kon - IME-USP.
%   \item Prof. Dr. José Carlos Maldonado - ICMC-USP.
%   \end{itemize}
%	  
%	\end{quote}
% \pagebreak

\pagenumbering{roman} % começamos a numerar

% ---------------------------------------------------------------------------- %
% Agradecimentos
\chapter*{Agradecimentos}

Este trabalho contou com o apoio do projeto Qualipso \cite{Qualipso}.

Gostaria de agradecer ao Christian Reis por sua ajuda, pelas
discussões interessantes e pelo apoio.
% TODO Completar

% ---------------------------------------------------------------------------- %
% Resumo
\chapter*{Resumo}

A relação entre métodos ágeis e software livre é, no mínimo,
indefinida. A princípio, as duas comunidades não parecem ter nenhuma
relação já que uma representa uma família de metodologias de
desenvolvimento de software e a outra, uma forma de licenciar código
fonte de um projeto. Com um pouco mais de observação percebe-se que as
comunidades compartilham diversas práticas e parece que as motivações
para aplicar tais práticas são semelhantes. Esse trabalho estuda essa
relação mais a fundo e apresenta semelhanças e diferenças entre as
duas comunidades. A partir disso, espera-se facilitar a identificação
das soluções de cada comunidade e contribuir com sugestões de
ferramentas e processos de desenvolvimento em ambos ambientes.

\noindent \textbf{Palavras-chave:} métodos ágeis, open source,
software livre

% ---------------------------------------------------------------------------- %
% Abstract
\chapter*{Abstract}

The relationship between agile methods and open source software is, at
least, undefined. At first glance, the two communities do not seem to
have any relationship since one represents a family of software
development methodologies and the other, a way to license a project's
source code. With a bit more observation one can notice that the
communities share several practices and appear to be motivated by the
same reasons. This work studies this relationship more deeply and
presents similarities and differences between the two
communities. Those results should help to identify the solutions of
each community and contribute with suggestions of development tools
and processes in both environments.

\noindent \textbf{Keywords:} agile methods, open source

% ---------------------------------------------------------------------------- %
% Sumário
\tableofcontents % imprime o sumário

% ---------------------------------------------------------------------------- %
% Listas: abreviaturas, símbolos, figuras e tabelas

\chapter{Lista de Abreviaturas}
\begin{tabular}{ll}
  SL       & Software Livre.\\
  OSS         & Software de Código Aberto (\emph{Open Source
    Software}).\\
  XP       & Programação Extrema (\emph{Extreme Programming}).\\
  FLOSS       & Software Gratuito, Livre e de Código Aberto
  (\emph{Free, Libre and Open Source Software}).\\
  TDD       & Desenvolvimento Dirigido por Teste
  (\emph{Test Driven Development}).\\
  BDD       & Desenvolvimento Dirigido por Comportamento
  (\emph{Behaviour Driven Development}).\\
  IRC       & Papo Retransmitido pela Internet (\emph{Internet Relay
    Chat}).\\
  FISL       & Fórum Internacional de Software Livre\\
  API       & Interface de Programação da Aplicação (\emph{Application
    Programming Interface}).\\
  OMM       & Modelo de Maturidade para Software Livre (\emph{Open
    Source Maturity Model}).\\
  CMM       & Modelo de Maturidade de Capabilidade (\emph{Capability
    Maturity Model}).\\
  SEI       & Instituto de Engenharia de Software (\emph{Software
    Engineering Institute}).\\
\end{tabular}

% \chapter{Lista de Símbolos}
% \begin{tabular}{ll}
%		$\omega$    & Freqüência angular.\\
%\end{tabular}

\listoffigures % lista de Figuras
%\listoftables % lista de Tabelas

% ---------------------------------------------------------------------------- %
% Capítulos
\mainmatter
% cabecalho para as páginas do 'mainmatter'
\fancyhead[RE,LO]{\thesection}

% \singlespacing % espaçamento simples
\onehalfspacing % espaçamento um e meio
% \doublespacing % espaçamento duplo

\input cap-introducao % associado ao arquivo: 'cap-introducao.tex'
\input cap-escopo % associado ao arquivo: 'cap-escopo.tex'
\input cap-semelhancas % associado ao arquivo: 'cap-semelhancas.tex'
\input cap-pesquisas % associado ao arquivo: 'cap-pesquisas.tex'
\input cap-diferencas % associado ao arquivo: 'cap-diferencas.tex'
\input cap-omm % associado ao arquivo: 'cap-omm.tex'
\input cap-conclusoes % associado ao arquivo: 'cap-conclusoes.tex'

% cabecalho para os apêndices
\renewcommand{\chaptermark}[1]{\markboth{\MakeUppercase{\appendixname\ \thechapter}} {\MakeUppercase{#1}} }
\fancyhead[RE,LO]{}
\appendix

\include{ape-pesquisaEA}      % associado ao arquivo: 'ape-pesquisaEA.tex'
\include{ape-pesquisaOS}      % associado ao arquivo: 'ape-pesquisaOS.tex'
\include{ape-pesquisaMA}      % associado ao arquivo: 'ape-pesquisaMA.tex'

% ---------------------------------------------------------------------------- %
% Bibliografia
\backmatter \singlespacing   % espaçamento simples

\bibliographystyle{alpha-ime}% citação bibliográfica alpha
\bibliography{bibliografia}  % associado ao arquivo: 'bibliografia.bib'

% ---------------------------------------------------------------------------- %
% Índice remissivo
%\index{TBP|see{periodicidade região codificante}}

%\printindex   % imprime o índice remissivo no documento 

\end{document}

