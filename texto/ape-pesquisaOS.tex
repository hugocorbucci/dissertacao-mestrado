\chapter{Pesquisa para colaboradores de Software Livre}
\label{ape:OS}

\singlespacing

A pesquisa abaixo é uma versão traduzida para o Português e adaptada
para impressão da pesquisa disponibilizada em
\url{http://www.ime.usp.br/~corbucci/floss-survey.html} do dia 28 de
Julho de 2009 até o dia 1$^o$ de Novembro de 2009.

\begin{enumerate}
\item País de residência: \verb=_________________=

\item Ano de nascimento: \verb=______=

\item Número de projetos livres com os quais já contribuiu:
  \verb=____=

\item Ano da primeira contribuição num projeto livre: \verb=______=

\item Nome do principal projeto livre para o qual contribui (ou
  contribuiu): \verb= ______________=

\item Ano da primeira contribuição para esse projeto: \verb=______=

\item Principal papel (escolha apenas 1) nesse projeto:
  \begin{itemize}
  \item[( )] Mantenedor
  \item[( )] \textit{Commiter}
  \item[( )] Programador
  \item[( )] Testador
  \item[( )] Documentador
  \item[( )] Relator de bugs/Descritor de requisitos
  \item[( )] Usuário
  \item[( )] Outro: \verb=_________________=
  \end{itemize}

\item Recebe (ou recebeu) algum rendimento por suas contribuições em
  projetos livres?
  \begin{itemize}
  \item[( )] Sim
  \item[( )] Não
  \end{itemize}

\item Se sim, é (ou foi) sua principal fonte de rendimentos?
  \begin{itemize}
  \item[( )] Sim
  \item[( )] Não
  \end{itemize}

\item Se não em alguma das duas anteriores, sua principal fonte de
  renda está ligada com Tecnologia da Informação?
  \begin{itemize}
  \item[( )] Sim
  \item[( )] Não
  \end{itemize}

\item Quantas pessoas trabalham (ou trabalhavam) com você (escolha
  apenas 1) no seu principal projeto livre?
  \begin{itemize}
  \item[( )] 0
  \item[( )] 1 a 5
  \item[( )] 6 a 10
  \item[( )] 11 a 50
  \item[( )] Mais de 50
  \end{itemize}

\item Qual é (ou foi) o principal canal de comunicação (escolha apenas
  1) entre essa equipe?
  \begin{itemize}
  \item[( )] Face a face
  \item[( )] Site na Internet
  \item[( )] Lista de correio eletrônico
  \item[( )] Ferramenta de rastreamento de problemas
  \item[( )] IRC (\textit{Internet Relay Chat} ou Papo Retransmitido
    pela Internet)
  \item[( )] Mensagens Instantâneas
  \item[( )] Correios eletrônicos individuais
  \item[( )] Voz sobre IP (Skype, Ekiga, iChat, etc...)
  \item[( )] Nenhum
  \item[( )] Outro: \verb=_____________=
  \end{itemize}

\item Como você avalia a qualidade de communicação da equipe?

  Péssima \verb=---------------------------------------= Ótima

\item Qual é (ou foi) o principal canal de comunicação (escolha apenas
  1) entre a equipe e os usuários?
  \begin{itemize}
  \item[( )] Face a face
  \item[( )] Site na Internet
  \item[( )] Lista de correio eletrônico
  \item[( )] Ferramenta de rastreamento de problemas
  \item[( )] IRC (\textit{Internet Relay Chat} ou Papo Retransmitido
    pela Internet)
  \item[( )] Mensagens Instantâneas
  \item[( )] Correios eletrônicos individuais
  \item[( )] Voz sobre IP (Skype, Ekiga, iChat, etc...)
  \item[( )] Nenhum
  \item[( )] Outro: \verb=_____________=
  \end{itemize}

\item Como você avalia a qualidade de communicação entre a equipe e os
  usuários?

  Péssima \verb=---------------------------------------= Ótima

\item Quanto esforço você investe (ou investiu) para manter as
  informações do projeto atualizadas?

  Nenhum \verb=---------------------------------------= Enorme

\item Quais das seguintes ferramentas seus projeto já utiliza? Marque
  todas que já utiliza ou utilizou.
  \begin{itemize}
  \item[( )] Mensagem ou Correio Eletrônico automático em caso de
    falha na montagem automática do projeto
  \item[( )] Estado dinâmico da versão em desenvolvimento a partir da
    ferramenta de rastreamento de problemas
  \item[( )] Gerenciamento da ferramenta de rastreamento de problemas
    a partir dos logs de commit do repositório de código
  \item[( )] Lançamento de nova versão a partir de um click no site do
    projeto
  \item[( )] Geração e atualização automática de gráficos de métricas
    a partir do repositório de código
  \item[( )] Ordenação colaborativa da prioridade dos problemas a
    serem endereçados pela equipe de desenvolvimento
  \item[( )] Linha do tempo no site do projeto ligada ao repositório
    de forma a facilitar análises em retrospectivas
  \item[( )] Um robô nos canais de communicação da equipe facilitar a
    comunicação assíncrona
  \end{itemize}

\item Ordene as seguintes ferramentas da que mais reduz (ou reduziria)
  o esforço gasto em comunicação no seu projeto para a que menos reduz
  (ou reduziria). Dê uma nota de 1 a 8 sendo que 1 é a ferramenta mais
  importante e 8 é a menos importante.
  \begin{itemize}
  \item[( )] Mensagem ou Correio Eletrônico automático em caso de
    falha na montagem automática do projeto
  \item[( )] Estado dinâmico da versão em desenvolvimento a partir da
    ferramenta de rastreamento de problemas
  \item[( )] Gerenciamento da ferramenta de rastreamento de problemas
    a partir dos logs de commit do repositório de código
  \item[( )] Lançamento de nova versão a partir de um click no site do
    projeto
  \item[( )] Geração e atualização automática de gráficos de métricas
    a partir do repositório de código
  \item[( )] Ordenação colaborativa da prioridade dos problemas a
    serem endereçados pela equipe de desenvolvimento
  \item[( )] Linha do tempo no site do projeto ligada ao repositório
    de forma a facilitar análises em retrospectivas
  \item[( )] Um robô nos canais de communicação da equipe de forma a
    facilitar a comunicação assíncrona
  \end{itemize}
\end{enumerate}
