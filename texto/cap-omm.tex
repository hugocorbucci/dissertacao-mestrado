%% ------------------------------------------------------------------------- %%
\chapter{Métodos Ágeis Abertos para o OMM}
\label{cap:omm}

O objetivo do projeto QualiPSo\footnote{\url{http://www.qualipso.org}
  - Último acesso em 27/08/2010} é de aumentar a confiabilidade da
indústria e a qualidade dos sistemas livres existentes e futuros. Para
atingir esse objetivo, o projeto conta com 10 grandes áreas de
trabalho. Uma dessas áreas é relacionado à confiabilidade do processo
usado no desenvolvimento de projetos livres.

Desde o início, o projeto abraçou o fato de que não poderia jamais
forçar uma forma de trabalho a comunidades livres. Por isso, a
abordagem usada para aumentar essa confiabilidade foi estabelecer uma
forma de avaliar a qualidade do processo usado por um determinado
projeto livre. Sendo assim, o projeto procurou elaborar um selo que
pudesse ser dado às comunidades que estivesse de acordo com um modelo
de processo confiável.

Porém, o contexto de projetos livre difere (como foi apresentado
anteriormente) do contexto para ambientes empresariais comuns. Por
isso, modelos de avaliação de processos estabelecidos na indústria não
são adequados para ambientes livres. Por isso, decidiu-se elaborar o
modelo de maturidade para software livre do Qualipso (\textit{QualiPSo
  Opensource Maturity Model} - OMM).

A seção \ref{sec:o-que-eh-omm} apresenta mais detalhes da origem do
OMM e de sua constituição. Em seguida, a seção \ref{sec:xp-em-omm}
apresenta como programação extrema pode ser mapeada para o OMM e quais
são os pontos não tratados. Por fim, a seção
\ref{sec:openagile-em-omm} apresenta uma sugestão de práticas
complementares para permitir a aprovação no OMM.

\section{Origem e descrição do OMM}
\label{sec:o-que-eh-omm}

O OMM se baseia na ideia de que a indústria confia em certificados de
qualidade. Padrões como o selo
ISO9001\footnote{\url{http://www.iso.org/} - Último acesso em
  27/08/2010} ou como o Modelo de Maturidade de Capabilidade
(\textit{Capability Maturity Model} - CMM) do Instituto de Engenharia
de Software ( \textit{Software Engineering Institute} - SEI)
\footnote{\url{http://www.sei.cmu.edu/cmmi} - Último acesso em
  27/08/2010} são constituídos de documentos que descrevem uma lista
de exigências que precisam ser cumpridas pelo processo das empresas
que esperaraem obter o selo.

Como projetos livres raramente beneficiam de uma infraestrutura física
ou organizacional, é muito difícil avaliar esses processos de acordo
com esses padrões da indústria. Por isso, o QualiPSo propôs trabalhar
num modelo baseado no CMM mas que pudesse ser usado não apenas para
empresas que incluem software livre em suas soluções mas também pelas
comunidades livres ao redor do mundo. Desse fato, decorre uma nota
importante sobre o OMM. O modelo todo foi pensado para que fosse
simples e fácil de usar pelos vários níveis organizacionais existentes
no ambiente de softwre livre.

\section{Um mapeamento de Programação Extrema para o OMM}
\label{sec:xp-em-omm}


\section{Um proposta ágil compatível com o OMM}
\label{sec:openagile-em-omm}

