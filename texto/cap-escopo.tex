%% ------------------------------------------------------------------------- %%
\chapter{Escopo}
\label{cap:escopo}

Para poder falar sobre software livre e métodos ágeis, é necessário,
primeiro, definir o que deve ser entendido por estes conceitos. A
Seção \ref{sec:agile-def} apresenta o escopo de projetos considerados
ágeis enquanto o escopo de projetos de software livre, mais
controversa, é apresentado na Seção \ref{sec:os-def}.

\section{Escopo de Métodos Ágeis abordados}
\label{sec:agile-def}

Este trabalho considerará como método ágil qualquer método de
engenharia de software que siga os princípios do manifesto ágil
\cite{AgileManifesto}. O interesse principal do trabalho está ligado
aos métodos mais conhecidos, como Programação Extrema (XP)
\cite{XP02}, Scrum \cite{Schwaber2004} e a família Crystal
\cite{Cockburn2002}. Mas também serão mencionadas algumas ideias
relacionadas à ``filosofia'' \emph{Lean} \cite{Ohno1998} e sua
aplicação ao desenvolvimento de software \cite{Poppendieck2005}.

\section{Escopo da comunidade de Software Livre abordada}
\label{sec:os-def}

Os termos ``Software de Código Aberto'' e ``Software Livre'' serão
considerados os mesmos neste trabalho apesar de terem diferenças
importantes em seus contextos específicos \cite[Ch. 1, Free Versus
Open source]{Fogel2005}. Ao longo do trabalho, quando se falar de
projetos de software livre serão considerados projetos cujo código
fonte estiver disponível e puder ser modificado por qualquer pessoa
com o conhecimento técnico necessário sem consentimento prévio do
autor original e sem encargos.

Outra restrição será de que projetos de software livre iniciados e
controlados por uma única empresa não serão tratados nesse
trabalho. Isto porque projetos controlados por empresas onde seja
disponibilizado o código fonte e/ou sejam aceitas colaborações
externas podem ser desenvolvidos com qualquer método de engenharia de
software. Basta que a empresa obrigue seus funcionários a seguir
determinado método. Alguns métodos incluem práticas que atraem
contribuições externas, outros distribuem apenas o trabalho escolhido
aos membros da equipe. De qualquer forma, a empresa controla sua
própria equipe e mantém o controle sobre o desenvolvimento
independentemente de colaborações.

No entanto, projetos livres baseados em comunidades de empresas podem
ser caracterizados como projetos de software livre se não existir um
contrato que force cada empresa a dedicar uma determinada quantidade
de horas de trabalho para o projeto. Caem neste caso o Eclipse com a
\emph{Eclipse Foundation} que, apesar de ter sido iniciado pela IBM,
agrega diversas empresas parceiras e o Java com o \emph{Java Community
  Process} que permite que a comunidade tome decisões sobre o
desenvolvimento da linguagem apesar da Sun ser proprietária da
marca. Esses contextos se assemelham ao de um desenvolvedor ou uma
equipe central trabalhando em conjunto com indivíduos ou equipes de
forma voluntária e, por isso, podem ser considerados software livre
conforme o contexto deste texto.

Trabalhos acadêmicos cujo código é liberado como software livre podem
entrar no escopo desse trabalho caso sigam um modelo distribuído com
contribuições não controladas. No caso de equipes completamente
controladas, o caso é muito semelhante ao da empresa que controla seus
funcionários.

Considerando esta definição, é importante caracterizar as pessoas
envolvidas em tais projetos. Em 2002, o \emph{FLOSS Project}
\cite{FlossProject} publicou um relatório sobre uma pesquisa realizada
com contribuidores de projetos de software livre. Os dados coletados
mostram que 78.77\% dos contribuidores têm emprego (pergunta 42) e que
apenas 50.82\% da comunidade de software livre são programadores
enquanto 24.76\% não ganham a maioria de suas rendas com
desenvolvimento de software (pergunta 10) \cite{FlossStats}. Além
desses resultados, a pesquisa apresenta o fato de 78.78\% dos
colaboradores considerarem suas tarefas em projetos livres mais
prazerosas (pergunta 22.2) do que suas atividades regulares. 42.30\%
também consideram seus projetos de software livre mais organizados que
seus projetos profissionais (pergunta 22.4).  Esses sentimentos sobre
as atividades dos contribuidores de software livre podem estar ligados
à liberdade no sistema de desenvolvimento dos projetos que, em geral,
não possuem nenhum processo pesado de desenvolvimento.

Por pesado, entende-se um processo no qual é muito importante
documentar rigorosamente as decisões tomadas e a maneira na qual
atingiu-se essa decisão. Tipicamente, estes processos contam com uma
importante fase de planejamento de forma a garantir que os documentos
que explicam a tomada de decisão sejam úteis e apresentem análises das
várias possibilidades. O termo ``processo pesado'' veio da comunidade
de métodos ágeis nos tempos em que eles eram ditos leves. As palavras
eram usadas como alusão à quantidade de tarefas obrigatórias para
chegar à implementação em processos pesados.

% TODO Já abordou outro assunto
Outra pesquisa \cite{Reis2003} apontou que 74\% dos projetos de
software livre tem equipes com até 5 pessoas e que 62\% dos
contribuidores nunca se conheceram fisicamente. Portanto, é crítico
para esses projetos que o processo de desenvolvimento esteja adequado
a essas características e não se torne um fardo para o trabalho
voluntário.

Tendo deixado claro o que será considerado um método ágil e um projeto
de software livre neste texto, o capítulo seguinte (Capítulo
\ref{cap:semelhancas}) aborda as semelhanças entre o desenvolvimento
de Software Livre e os conceitos de Métodos Ágeis.
