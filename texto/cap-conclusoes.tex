%% ------------------------------------------------------------------------- %%
\chapter{Conclusões}
\label{cap:conclusoes}

% TODO Escrever uma conclusão

%------------------------------------------------------
\section{Considerações finais}

Neste trabalho, foram mostradas diversas evidências de que a sinergia
entre métodos ágeis e projetos de software livre pode beneficiar o
processo de desenvolvimento de projetos livres e ajudar os atuais
métodos ágeis a lidar com dificuldades conhecidas. Alguns projetos de
software livre já adotam algumas técnicas ágeis para serem mais
eficientes com relação aos pedidos dos usuários mas a descrição de um
método ágil que considere todos os fatores do Software Livre
provavelmente aumentaria a adoção das práticas nessas comunidades. Por
outro lado, resolver o problema é um desafio que poderia consolidar
métodos ágeis em ambientes distribuídos com o apoio de uma grande
comunidade de usuários.

Como parte desse trabalho, duas pesquisas estão planejadas. A primeira
deve ser em sites de apoio a projetos livres para entender qual o grau
de envolvimento com métodos ágeis atualmente existente na comunidade
de software livre e quais as dificuldades encontradas por estas
pessoas para adotarem mais práticas ágeis. A outra deve ser divulgada
durante a Agile 2009 que será realizada no fim de Agosto. Esta
pesquisa procura avaliar qual o envolvimento da comunidade de métodos
ágeis em projetos de software livre. Ambas pesquisas serão usadas para
prover um maior entendimento das interações entre ambas comunidades e
como melhorá-las.


