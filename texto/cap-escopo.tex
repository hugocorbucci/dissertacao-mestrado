%% ------------------------------------------------------------------------- %%
\chapter{Escopo}
\label{cap:escopo}

Para poder falar sobre software livre e métodos ágeis, é necessário,
primeiro, definir o que deve ser entendido por estes conceitos. A
Seção~\ref{sec:agile-def} apresenta o escopo de projetos considerados
ágeis enquanto o escopo de projetos de software livre, mais
complexo, é apresentado na Seção~\ref{sec:os-def}.

% TODO Talvez falar mais das comunidades, caracteristicas e ideias

\section{Escopo de métodos ágeis abordados}
\label{sec:agile-def}

Este trabalho considerará como método ágil qualquer método de
engenharia de software que siga os princípios do manifesto
ágil~\cite{AgileManifesto}. O interesse principal do trabalho está
ligado aos métodos mais conhecidos, como Programação Extrema
(\emph{eXtreme Programming} - XP)~\cite{XP02},
Scrum~\cite{Schwaber2004} e a família Crystal~\cite{Cockburn2002}. Mas
também serão mencionadas algumas ideias relacionadas à ``filosofia''
\emph{Lean}~\cite{Ohno1998} e sua aplicação ao desenvolvimento de
software~\cite{Poppendieck2005}.

\section{Escopo da comunidade de software livre abordada}
\label{sec:os-def}

Os termos ``Software de Código Aberto'' e ``Software Livre'' serão
considerados os mesmos neste trabalho apesar de terem diferenças
importantes em seus contextos específicos~\cite[Ch. 1, Free Versus
Open source]{Fogel2005}. Ao longo do trabalho, quando se falar de
projetos de software livre serão considerados projetos cujo código
fonte estiver disponível e puder ser modificado por qualquer pessoa
com o conhecimento técnico necessário. Não deve ser necessário nenhum
consentimento adicional (além da licença) por parte do autor original
e não pode ser cobrado nenhum encargo para realizar a mudança.

Outra restrição será de que projetos de software livre iniciados e
controlados por uma única empresa não serão tratados nesse
trabalho. Isto porque projetos controlados por empresas onde seja
disponibilizado o código fonte e/ou sejam aceitas colaborações
externas podem ser desenvolvidos com qualquer processo de engenharia
de software. Basta que a empresa obrigue seus funcionários a seguir
determinadas instruções. Alguns métodos incluem práticas que atraem
contribuições externas, outros distribuem apenas o trabalho escolhido
aos membros da equipe. De qualquer forma, a empresa controla sua
própria equipe e mantém o controle sobre o desenvolvimento
independentemente de colaborações.

Vale notar que, pela natureza de projetos livres, uma empresa só
consegue manter controle sobre seus funcionários e, caso não dê a
devida atenção e liberdade à comunidade, esta pode tomar as rédeas em
um novo projeto baseado no original.  É exatamente este o cenário que
aconteceu com o
OpenOffice.org\footnote{\url{http://www.openoffice.org} -- Último
  acesso em 24/01/2011} após a
Oracle\footnote{\url{http://www.oracle.com} -- Último acesso em
  24/01/2011} forçar algumas políticas de adições proprietárias ao
projeto\footnote{\url{http://www.lehsys.com/2010/11/oracle-triggers-mass-defection-of-openoffice-developers-to-libreoffice}
  -- Último acesso em 24/01/2011}. A comunidade não aceitou a decisão
e criou uma nova versão do projeto chamada
LibreOffice\footnote{\url{http://www.libreoffice.org} -- Último acesso
  em 24/01/2011} que não está sujeita às regras ditadas pela empresa.

No entanto, projetos livres baseados em comunidades de empresas podem
ser caracterizados como projetos de software livre se não existir um
contrato que force cada empresa a dedicar uma determinada quantidade
de horas de trabalho para o projeto. Entram neste caso o
Eclipse\footnote{\url{http://www.eclipse.org} -- Último acesso em
  24/01/2011} com a \emph{Eclipse Foundation} que, apesar de ter sido
iniciado pela IBM\footnote{\url{http://www.ibm.com} -- Último acesso
  em 24/01/2011}, agrega diversas empresas parceiras e o
Java\footnote{\url{http://www.java.com} -- Último acesso em
  24/01/2011} com o \emph{Java Community Process} que permite que a
comunidade tome decisões sobre o desenvolvimento da linguagem apesar
da Oracle ser proprietária da marca. Esses contextos se assemelham ao
de um desenvolvedor ou uma equipe central trabalhando em conjunto com
indivíduos ou equipes de forma voluntária e, por isso, podem ser
considerados software livre conforme o contexto deste texto.

Trabalhos acadêmicos cujo código é liberado como software livre podem
entrar no escopo desse trabalho caso sigam um modelo distribuído com
contribuições não controladas. No caso de equipes completamente
controladas, o caso é muito semelhante ao da empresa que controla seus
funcionários e, portanto, não será tratado.

A próxima Seção (Seção~\ref{subsec:caracterizacao}) apresenta alguns
estudos que traçam o perfil de contribuidores desse tipo de projetos
livres.

% TODO pergunta deveria ser maiusculo se for p/ numero

\subsection{Caracterização dos contribuidores de software livre}
\label{subsec:caracterizacao}

Considerando o escopo definido na Seção anterior
(Seção~\ref{sec:os-def}), é importante caracterizar as pessoas
envolvidas em tais projetos. Em 2002, o \emph{FLOSS
  Project}~\cite{FlossProject} publicou um relatório sobre uma
pesquisa realizada com contribuidores de projetos de software
livre. Os dados coletados mostram que 79\% dos contribuidores tem
emprego (Pergunta~42) e que apenas 51\% da comunidade de software
livre são programadores enquanto 25\% não ganham a maioria de suas
rendas com desenvolvimento de software
(Pergunta~10)~\cite{FlossStats}. Além desses resultados, a pesquisa
apresenta o fato de 79\% dos colaboradores considerarem suas tarefas
em projetos livres mais prazerosas (Pergunta~22.2) do que suas
atividades regulares. 42\% também consideram seus projetos de software
livre mais organizados que seus projetos profissionais
(Pergunta~22.4).  Esses sentimentos sobre as atividades dos
contribuidores de software livre podem estar ligados à liberdade no
sistema de desenvolvimento dos projetos que, em geral, não possuem
nenhum processo pesado de desenvolvimento.

Por processo pesado de desenvolvimento, entende-se um processo no qual
é muito importante documentar rigorosamente as decisões tomadas e a
maneira na qual atingiu-se essa decisão. Tipicamente, estes processos
contam com uma importante fase de planejamento de forma a garantir que
os documentos que explicam a tomada de decisão sejam úteis e
apresentem análises das várias possibilidades. O termo ``processo
pesado'' veio da comunidade de métodos ágeis que, em contrapartida,
utilizava processos leves. As palavras eram usadas como alusão à
quantidade de tarefas obrigatórias para chegar à implementação em
processos pesados.

Outra pesquisa~\cite{Reis2003} apontou que 74\% dos projetos de
software livre tem equipes com até 5 pessoas e que 62\% dos
contribuidores nunca se conheceram fisicamente. Portanto, é crítico
para esses projetos que o processo de desenvolvimento esteja adequado
a essas características e não se torne um fardo para o trabalho
voluntário.

Tendo deixado claro o que será considerado um método ágil e um projeto
de software livre neste texto, o capítulo seguinte
(Capítulo~\ref{cap:semelhancas}) aborda as semelhanças entre o
desenvolvimento de software livre e os conceitos de métodos ágeis.
