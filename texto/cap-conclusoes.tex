%% ------------------------------------------------------------------------- %%
\chapter{Conclusões}
\label{cap:conclusoes}

Existem muitas semelhanças nos valores da comunidades de métodos ágeis
e de software livre. Em ambos os casos, alguns dos fundadores de cada
comunidade diziam ter iniciado seus respectivos movimentos para fugir
do estado da indústria de software em suas épocas. Raymond afirma que
entrou no movimento de software livre para nunca mais precisar se
submeter às pressões da indústria de software que o impedia de
desenvolver código decentemente. Do outro lado, os autores do
manifesto ágil afirmam que escreveram o manifesto para impulsionar a
indústria de desenvolvimento a mudar suas formas de trabalho.

Essas vontades iniciais semelhantes deixaram rastros importantes na
forma de trabalho existente. Uma forte confiança nos indivíduos e suas
capacidades de resolverem seus próprios problemas, uma minimização da
importância de processos, ferramentas ou documentos e uma atenção
redobrada à qualidade do produto final do desenvolvimento.

Apesar dessas raízes e valores comuns, os contextos ótimos e soluções
encontradas para os problemas decorrentes desses contextos são BEM
diferentes. Comunidades de software livre são naturalmente
distribuídas, envolvem pouquíssimas relações pessoais face a face,
contam com o poder da falha e redundância e se atacam principalmente a
problemas que a afligem. Agilistas dão preferência para equipes
pequenas situadas no mesmo local, com uso extenso de comunicação
direta e informal e evolução contínua com atenção extrema para
minimizar o trabalho realizado.

Ambas propostas tiveram um certo êxito em seus respectivos nichos e,
com seu sucesso, começaram a explorar contextos diferentes, outros
problemas e até mesmo o simples crescimento dos problemas enfrentados
com eles. Com isso, métodos ágeis rapidamente chegaram a problemas que
exigiam equipes distribuídas e o uso ou evolução de sistemas
desenvolvidos por terceiros. Empresas começaram a procurar integrar e
evoluir projetos livres com equipes próprias e para projetos que
buscam resolver problemas fora da áurea dos problemas de
desenvolvimento tradicionalmente abordados.

Dessa forma, pareceu natural tentar aproximar ambas comunidades e
descobrir o que poderia dificultar essa união. Os resultados das
pesquisas apresentados no Capítulo~\ref{cap:pesquisas} indicaram que
as comunidades, de fato, são constituídas de pessoas diferentes com
costumes e atividades diferentes. No entanto, os problemas
identificados e as ferramentas apropriadas para tratá-los foram
impressionantemente semelhantes.

A dicotomia formada pela semelhança das soluções e a diferença nos
costumes e atividades levou a uma análise mais aprofundada dos
princípios que norteam os membros de cada comunidade. Esses
princípios, apresentados no Capítulo~\ref{cap:diferencas}, evidenciam
as diferenças de origens e contextos ótimos de cada comunidade
deixando, no entanto, algumas boas práticas comuns emergirem. Dessas
boas práticas, notam-se especialmente as práticas que valorizam os
indivíduos envolvidos e a confiança depositada neles.