% Arquivo LaTeX de exemplo de dissertação/tese a ser apresentados à CPG
% do IME-USP
% 
% Versão 3: Thu Dec 10 23:57:33 BRST 2009

%
% Criação: Jesús P. Mena-Chalco
% Revisão: Fábio Kon
%  
% Obs: Leia previamente o texto do arquivo README.txt
%

\documentclass[11pt,twoside,a4paper]{book}

% ---------------------------------------------------------------------------- %
% Pacotes 
\usepackage[T1]{fontenc}
\usepackage[brazil]{babel}
\usepackage[utf8]{inputenc}
\usepackage[pdftex]{graphicx}           % usamos arquivos pdf/png como figuras
\usepackage{pifont}
\usepackage{amsfonts}
\usepackage{amssymb} 
\usepackage{setspace}                   % espaçamento flexível
\usepackage[bf,small,compact]{titlesec} % cabeçalhos dos títulos: menores e compactos
\usepackage{indentfirst}                % indentação do primeiro parágrafo
\usepackage{subfigure}                  % uso de várias figuras numa só
\usepackage{makeidx}                    % índice remissivo
\usepackage[nottoc]{tocbibind}          % acrescentamos a bibliografia/indice/conteudo no Table of Contents
\usepackage{courier}                    % usa o Adobe Courier no lugar de Computer Modern Typewriter
\usepackage{type1cm}                    % fontes realmente escaláveis
\usepackage{listings}                   % para formatar código-fonte (ex. em Java)
\usepackage{setspace}
\usepackage{longtable}
\usepackage{multirow}
\usepackage{titletoc}
\usepackage{lscape}
\usepackage[fixlanguage]{babelbib}
\usepackage[font=small,format=plain,labelfont=bf,up,textfont=it,up]{caption}
\usepackage[usenames,svgnames,dvipsnames,table]{xcolor}
\usepackage[a4paper,top=2.54cm,bottom=2.0cm,left=2.0cm,right=2.54cm]{geometry}% margens
\usepackage{float}
\usepackage[pdftex,plainpages=false,pdfpagelabels,pagebackref,colorlinks=true,citecolor=black,linkcolor=black,urlcolor=black,filecolor=black,bookmarksopen=true]{hyperref} % links em preto
%\usepackage[pdftex,plainpages=false,pdfpagelabels,pagebackref,colorlinks=true,citecolor=DarkGreen,linkcolor=NavyBlue,urlcolor=DarkRed,filecolor=green,bookmarksopen=true]{hyperref} % links coloridos
\usepackage[all]{hypcap}                % soluciona o problema com o hyperref e capitulos
\usepackage[numbers,square,sort,nonamebreak,comma]{natbib}  % citação
                                % bibliográfica alpha (alpha-ime.bst)

\usepackage{type1cm}      % fontes realmente escaláveis
\fontsize{60}{62}\usefont{OT1}{cmr}{m}{n}{\selectfont}

% ---------------------------------------------------------------------------- %
\floatstyle{boxed}
\newfloat{caixa}{htb}{bx}
\restylefloat{table}
\floatname{caixa}{Caixa}

% ---------------------------------------------------------------------------- %
% headers similares oa TAOP de Donald E. Knuth
\usepackage{fancyhdr}
\pagestyle{fancy}
\fancyhf{}
\renewcommand{\chaptermark}[1]{\markboth{\MakeUppercase{#1}}{}}
\renewcommand{\sectionmark}[1]{\markright{\MakeUppercase{#1}}{}}
\renewcommand{\headrulewidth}{0pt}

% ---------------------------------------------------------------------------- %
\graphicspath{{./figuras/}}             % path das figuras (recomendável)
\frenchspacing                          % Arruma o espaço: id est (i.e.) e exempli gratia (e.g.) 
\urlstyle{same}                         % URL com o mesmo estilo do texto e nao mono-spaced
\makeindex                              % para o índice remissivo
\raggedbottom                           % para não permitir espaços extra no texto
\fontsize{60}{62}\usefont{OT1}{cmr}{m}{n}{\selectfont}
\cleardoublepage
\normalsize

% ---------------------------------------------------------------------------- %
% Opções de listing usados para o código fonte
% Ref: http://en.wikibooks.org/wiki/LaTeX/Packages/Listings
\lstset{ %
language=Java,                  % choose the language of the code
basicstyle=\footnotesize,       % the size of the fonts that are used for the code
numbers=left,                   % where to put the line-numbers
numberstyle=\footnotesize,      % the size of the fonts that are used for the line-numbers
stepnumber=1,                   % the step between two line-numbers. If it's 1 each line will be numbered
numbersep=5pt,                  % how far the line-numbers are from the code
showspaces=false,               % show spaces adding particular underscores
showstringspaces=false,         % underline spaces within strings
showtabs=false,                 % show tabs within strings adding particular underscores
frame=single,	                % adds a frame around the code
framerule=0.6pt,
tabsize=2,	                    % sets default tabsize to 2 spaces
captionpos=b,                   % sets the caption-position to bottom
breaklines=true,                % sets automatic line breaking
breakatwhitespace=false,        % sets if automatic breaks should only happen at whitespace
escapeinside={\%*}{*)},         % if you want to add a comment within your code
backgroundcolor=\color[rgb]{1.0,1.0,1.0}, % choose the background color.
rulecolor=\color[rgb]{0.8,0.8,0.8},
extendedchars=true,
xleftmargin=10pt,
xrightmargin=10pt,
framexleftmargin=10pt,
framexrightmargin=10pt
}

% ---------------------------------------------------------------------------- %
% Corpo do texto
\begin{document}
\frontmatter
% headers para as páginas do frontmatter
\fancyhead[RO]{{\footnotesize\rightmark}\hspace{2em}\thepage}
\setcounter{tocdepth}{2}
\fancyhead[LE]{\thepage\hspace{2em}\footnotesize{\leftmark}}
\fancyhead[RE,LO]{}
\fancyhead[RO]{{\footnotesize\rightmark}\hspace{2em}\thepage}

\onehalfspacing % espaçamento


% ---------------------------------------------------------------------------- %
% Capa
\thispagestyle{empty}
\begin{center}
  \vspace*{2.3cm}
  \textbf{\Large{Métodos ágeis e software livre:\\
      Um estudo da relação entre\\
      estas duas comunidades}}\\
	
  \vspace*{1.2cm} \Large{Hugo Corbucci}
    
  \vskip 2cm \textsc{
    Dissertação apresentada\\[-0.25cm]
    ao\\[-0.25cm]
    Instituto de Matemática e Estatística\\[-0.25cm]
    da\\[-0.25cm]
    Universidade de São Paulo}
    
  \vskip 1.5cm
  Programa: Mestrado em Ciência da Computação\\
  Orientador: Prof. Dr. Alfredo Goldman

  \vskip 1cm \normalsize{Durante o desenvolvimento deste trabalho o
    autor recebeu auxílio financeiro do projeto Qualipso}
	
  \vskip 0.5cm \normalsize{São Paulo, Fevereiro de 2011}
\end{center}

% ---------------------------------------------------------------------------- %
% Página de rosto (só para a versão final) \newpage
% \thispagestyle{empty}
%	\begin{center}
%   \vspace*{2.3 cm}
%   \textbf{\Large{Título do trabalho a ser apresentado à \\
%   CPG para a dissertação/tese}}\\
%   \vspace*{2 cm}
%	\end{center}
%
%	\vskip 2cm
%
%	\begin{flushright}
%   Este exemplar corresponde à redação\\
%   final da dissertação devidamente corrigida\\
%   e defendida por Hugo Corbucci\\
%   e aprovada pela Comissão Julgadora.  \vskip 2cm
%
%	\end{flushright}
%	\vskip 4.2cm
%
%	\begin{quote}
%   \noindent Banca Examinadora:
%	
%   \begin{itemize}
%   \item Prof. Dr. Alfredo Goldman (orientador) - IME-USP.
%   \item Prof. Dr. Fabio Kon - IME-USP.
%   \item Prof. Dr. José Carlos Maldonado - ICMC-USP.
%   \end{itemize}
%	  
%	\end{quote}
% \pagebreak

\pagenumbering{roman} % começamos a numerar

% ---------------------------------------------------------------------------- %
% Agradecimentos
\chapter*{Agradecimentos}

Este trabalho contou com o apoio do projeto Qualipso \cite{Qualipso}.

Gostaria de agradecer ao Christian Reis por sua ajuda, pelas
discussões interessantes e pelas ideias. Aos meus pais, pela
confiança, amor e apoio. À Mariana Bravo pelo apoio, companhia e
ajudas constantes ao longo de todo esse tempo. Ao Danilo Sato e Daniel
Cordeiro pelas revisões e críticas ao texto além das excelentes
discussões. Ao Alexandre Freire e ao Fernando Freire pela sociedade,
oportunidades, longas discussões e sessões de programação. Ao Rafael
Prikladnicki, Daniel Cukier e Mauricio Aniche pelas oportunidades
oferecidas e revisões ao texto.

Aos alunos do Laboratório de Programação Extrema de 2007 a 2010 pelas
experiências proporcionadas. Aos amigos que sempre ofereceram
diversão, alegria e felicidade.  Aos outros membros da AgilCoop pelas
experiências, ideias, conversas e ajudas.

Por fim, mas muito importante, ao Prof. Dr. Alfredo Goldman pela
orientação, conversas, ajudas, apoio e amizade.

% ---------------------------------------------------------------------------- %
% Resumo
\chapter*{Resumo}

A relação entre métodos ágeis e software livre não é clara. A
princípio, os dois assuntos não parecem ter nenhuma relação já que
tratam de conceitos diferentes: uma família de metodologias de
desenvolvimento de software e uma forma de licenciar código fonte de
um projeto. No entanto, as pessoas envolvidas nos dois movimentos
formam comunidades cujo recente sucesso tem surpreendido a indústria
de software. Observando com um pouco mais de cuidado, percebe-se que
as comunidades compartilham diversas práticas e, aparentemente, as
motivações para aplicar tais práticas são semelhantes. Esse trabalho
estuda essa relação mais a fundo e apresenta semelhanças e diferenças
entre as duas comunidades. A partir disso, espera-se facilitar a
identificação das soluções de cada comunidade e contribuir com
sugestões de ferramentas e processos de desenvolvimento em ambos
ambientes. Em especial, para equipes que queiram desenvolver projetos
livres de qualidade, o trabalho apresenta uma análise da Programação
Extrema, do ponto de vista de um modelo de maturidade para ambientes
de software livre, o Modelo de Maturidade Aberto (OMM) do projeto
QualiPSo. Essa análise deve permitir uma adequação a alguns requisitos
básicos de qualidade mantendo características e formas de trabalho
comuns às comunidades livres.

\noindent \textbf{Palavras-chave:} métodos ágeis, open source,
software livre, modelo de maturidade aberto, omm, programação extrema,
xp

% ---------------------------------------------------------------------------- %
% Abstract
\chapter*{Abstract}

The relationship between agile methods and open source software is
unclear. At first glance, the two subjects do not seem to have any
relationship since they address different concepts: a family of
software development methodologies and a way to license a project's
source code. However, people involved in both movements form
communities that share several practices and appear to be motivated by
the same reasons. This work studies this relationship more deeply and
presents similarities and differences between the two
communities. This result should help to identify the solutions of each
community and contribute with suggestions of development tools and
processes in both environments. In particular, for teams wishing to
develop quality open source projects, this work presents an analysis
of eXtreme Programming, from the point of view of a maturity model
aimed at open source software environments, QualiPSo's Open Maturity
Model (OMM). This analysis should allow a team to fulfill basic
quality requirements while still keeping work characteristics and
behaviors common within free software communities.

\noindent \textbf{Keywords:} agile methods, open source, free
software, open maturity model, omm, extreme programming, xp

% ---------------------------------------------------------------------------- %
% Sumário
\tableofcontents % imprime o sumário

% ---------------------------------------------------------------------------- %
% Listas: abreviaturas, símbolos, figuras e tabelas

\chapter{Lista de Abreviaturas}
\begin{tabular}{lp{14cm}}
  SL       & Software Livre.\\
  OSS         & Software de Código Aberto (\emph{Open Source
    Software}).\\
  XP       & Programação Extrema (\emph{Extreme Programming}).\\
  FLOSS       & Software Gratuito, Livre e de Código Aberto
  (\emph{Free, Libre and Open Source Software}).\\
  OOPSLA       & Linguagens de Programação, Sistemas e Aplicações
  Orientadas a Objetos (\emph{Object-Oriented Programming Languages, Systems and Applications}).\\
  TDD       & Desenvolvimento Dirigido por Teste
  (\emph{Test Driven Development}).\\
  BDD       & Desenvolvimento Dirigido por Comportamento
  (\emph{Behaviour Driven Development}).\\
  IRC       & Papo Retransmitido pela Internet (\emph{Internet Relay
    Chat}).\\
  FISL       & Fórum Internacional de Software Livre.\\
  API       & Interface de Programação da Aplicação (\emph{Application
    Programming Interface}).\\
  OMM       & Modelo de Maturidade para Software Livre (\emph{Open
    Source Maturity Model}).\\
  CMM       & Modelo de Maturidade e de Capacidade (\emph{Capability
    Maturity Model}).\\
  SEI       & Instituto de Engenharia de Software (\emph{Software
    Engineering Institute}).\\
  GQM       & Objetivo Pergunta Métrica (\emph{Goal Question Metric}).\\
  TI       & Tecnologia da Informação.\\
\end{tabular}

% \chapter{Lista de Símbolos}
% \begin{tabular}{ll}
%		$\omega$    & Freqüência angular.\\
%\end{tabular}

\listoffigures % lista de Figuras
% \listoftables % lista de Tabelas

% ---------------------------------------------------------------------------- %
% Capítulos
\mainmatter
% cabecalho para as páginas do 'mainmatter'
\fancyhead[RE,LO]{\thesection}

% \singlespacing % espaçamento simples
\onehalfspacing % espaçamento um e meio
% \doublespacing % espaçamento duplo

% TODO Revisar M.A. e S.L. p/ ficar minusculo
% TODO Revisar Capitulo, Secao etc para ficar minusculo se nao for o
% tal (com numero)
% TODO Gráficos ficam ruins na impressão. Cores não são != suficientes
% TODO Revisar traduções do Fabio (http://www.ime.usp.br/~kon/ResearchStudents/traducao.html)

\input cap-introducao % associado ao arquivo: 'cap-introducao.tex'
\input cap-escopo % associado ao arquivo: 'cap-escopo.tex'
\input cap-semelhancas % associado ao arquivo: 'cap-semelhancas.tex'
\input cap-pesquisas % associado ao arquivo: 'cap-pesquisas.tex'
\input cap-diferencas % associado ao arquivo: 'cap-diferencas.tex'
\input cap-omm % associado ao arquivo: 'cap-omm.tex'
\input cap-conclusoes % associado ao arquivo: 'cap-conclusoes.tex'

% ---------------------------------------------------------------------------- %
% Bibliografia
\backmatter \singlespacing   % espaçamento simples

\bibliographystyle{alpha-ime}% citação bibliográfica alpha
\bibliography{bibliografia}  % associado ao arquivo: 'bibliografia.bib'

% ---------------------------------------------------------------------------- %
% Índice remissivo
%\index{TBP|see{periodicidade região codificante}}

%\printindex   % imprime o índice remissivo no documento 
% cabecalho para os apêndices

\renewcommand{\chaptermark}[1]{\markboth{\MakeUppercase{\appendixname\
      \thechapter}} {\MakeUppercase{#1}} } \fancyhead[RE,LO]{}
\appendix

\include{ape-pesquisaEA}      % associado ao arquivo: 'ape-pesquisaEA.tex'
\include{ape-pesquisaOS}      % associado ao arquivo: 'ape-pesquisaOS.tex'
\include{ape-pesquisaMA}      % associado ao arquivo: 'ape-pesquisaMA.tex'

\end{document}

