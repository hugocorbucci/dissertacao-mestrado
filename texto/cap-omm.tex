%% ------------------------------------------------------------------------- %%
\chapter{Métodos Ágeis abertos para o OMM}
\label{cap:omm}

O objetivo do projeto QualiPSo\footnote{\url{http://www.qualipso.org}
  - Último acesso em 27/08/2010} é de aumentar a confiabilidade da
indústria e a qualidade dos sistemas livres existentes e futuros. Para
atingir esse objetivo, o projeto conta com 10 grandes áreas de
trabalho. Uma dessas áreas corresponde à confiabilidade do processo
usado no desenvolvimento de projetos livres.

Desde o início, o projeto abraçou o fato de que não poderia jamais
forçar uma forma de trabalho a comunidades livres. Por isso, a
abordagem usada para aumentar essa confiabilidade foi estabelecer uma
forma de avaliar a qualidade do processo usado por um determinado
projeto livre. Sendo assim, o projeto procurou elaborar um selo que
pudesse ser dado às comunidades e às empresas que estivessem de acordo
com um modelo de processo confiável.

Porém, o contexto de projetos livres difere (como apresentado
anteriormente na Seção~\ref{sec:os-def}) do contexto para ambientes
empresariais comuns. Por isso, modelos de avaliação de processos
estabelecidos na indústria não são adequados para ambientes livres. Em
decorrência, decidiu-se elaborar o modelo de maturidade para software
livre do QualiPSo (\textit{QualiPSo Opensource Maturity Model} - OMM).

A Seção~\ref{sec:o-que-eh-omm} apresenta mais detalhes da origem do
OMM e de sua constituição. Em seguida, a Seção~\ref{sec:xp-em-omm}
apresenta como programação extrema pode ser mapeada para o OMM e quais
são os pontos não tratados. Por fim, a Seção~\ref{sec:sl+omm}
apresenta a situação atual de metodologias em comunidades livres e
como o OMM pode contribuir na modelagem de uma metodologia geral
nessas comunidades.

\section{Origem e descrição do OMM}
\label{sec:o-que-eh-omm}

O OMM se baseia na ideia de que, na indústria, certificados de
qualidade possuam boa aceitação. Padrões como o selo
ISO9001\footnote{\url{http://www.iso.org/} - Último acesso em
  27/08/2010} ou como o Modelo de Maturidade de Capabilidade
(\textit{Capability Maturity Model} - CMM) do Instituto de Engenharia
de Software (\textit{Software Engineering Institute} -
SEI)\footnote{\url{http://www.sei.cmu.edu/cmmi} - Último acesso em
  27/08/2010} são constituídos de documentos que descrevem uma lista
de exigências que precisam ser cumpridas nos processos das empresas
que esperarem obter o selo.

Como projetos livres raramente beneficiam de uma infra-estrutura
física ou organizacional, é muito difícil avaliar esses processos de
acordo com esses padrões da indústria. Por isso, o QualiPSo propôs
trabalhar num modelo baseado no CMM mas que pudesse ser usado não
apenas para empresas que incluem software livre em suas soluções mas
também pelas comunidades livres ao redor do mundo. Desse fato, decorre
uma nota importante sobre o OMM. O modelo todo foi pensado para que
fosse simples e fácil de usar pelos vários níveis organizacionais
existentes no ambiente de software livre.

A primeira fase de elaboração do OMM foi realizar um levantamento dos
chamados elementos de confiabilidade (\textit{Trusthworthy elements})
no contexto de software livre. Os elementos identificados formaram a
base do OMM para garantir que o processo avaliado não apresentasse
apenas qualidade e confiabilidade do ponto de vista comercial mas
também no contexto de comunidades livres.

Levantados esses elementos de confiabilidade, a equipe do OMM realizou
um mapeamento das áreas de qualidades avaliadas no CMM para
identificar quais elementos eram abordados e quais não eram. Os
principais elementos de confiabilidade que o CMM não aborda estão
relacionados aos problemas legais do uso de software, à reputação de
determinado projeto e do tamanho de sua comunidade.

No aspecto legal, as questões do licenciamento do código, da violação
de patentes e preservação de marcas são pontos importantíssimos para
permitir o uso de qualquer projeto livre numa organização
comercial. Na questão das contribuições, é importante tomar cuidado
com a questão dos direitos autorais para evitar problemas legais
relacionados ao licenciamento do código. Esses dois aspectos não são
tratados ou sequer abordados no CMM já que assume-me que, se uma
empresa utilizar código externo, esse código será obtido sob um
contrato estabelecido pela empresa com o proprietário do código. Nesse
caso, as preocupações são menores já que há um contrato explicitamente
assinado pelas partes envolvidas que rege a relação.

Por outro lado, o CMM aborda alguns aspectos que são importantes para
a confiabilidade de um projeto no contexto comercial. Muitos desses
aspectos estão ligados a exigências na quantidade e detalhamento de
documentos usados para inspeção e melhoria dentro da organização que
implementa o processo. A equipe do OMM selecionou todas as práticas
sugeridas pelo CMM no que diz respeito aos aspectos técnicos e apenas
algumas no aspecto gerencial que fazem sentido no contexto livre.

Graças a esse trabalho, o OMM foi formado com um misto de elementos de
confiabilidade vindos da comunidade de software livre com práticas
estabelecidas vindas do CMM. O modelo ainda optou por adotar uma
estrutura piramidal semelhante à do CMM na qual existem três níveis de
adequação sendo que o mais básico é base para os mais avançados que
sempre exigem todas as práticas do nível inferior e mais algumas.

A Figura~\ref{fig:piramide-omm} apresenta a divisão de níveis com a
lista de práticas (com nomes abreviados) que integra cada um dos
níveis do OMM. Além disso, o OMM propõe exigências diferentes de
acordo com o tipo de entidade que deseja ser avaliada para um
determinado nível. Isto é, algumas práticas são apenas recomendadas e
não obrigatórias para comunidades livres não representadas por uma
empresa. Dessa forma, quando o projeto não tem uma organização por
trás, os membros da comunidade só precisam realizar o que está no
alcance de uma comunidade para atingir um determinado nível.

\begin{figure}
  \centering
  \includegraphics[scale=0.4]{omm-levels}
  \caption{Pirâmide de elementos essenciais exigidos para cada um dos
    níveis do OMM}
  \label{fig:piramide-omm}
\end{figure}

As Tabelas~\ref{tab:omm-basic}, \ref{tab:omm-intermediate}
e~\ref{tab:omm-advanced} mostram os elementos de confiabilidade que
precisam ser abordados para se atingir os níveis básico, intermediário
e avançados do OMM.

\begin{table}
  \centering
  \begin{tabular}{|p{2cm}|p{13cm}|}
    \hline
    PDOC & Documentação do Produto (\emph{Product Documentation}) \\
    \hline
    STD & Uso de Padrões Estabelecidos e Adotados (\emph{Use of
      Established and Widespread Standards}) \\
    \hline
    QTP & Qualidade do Plano de Testes (\emph{Quality of Test Plan})
    \\
    \hline
    LCS & Licenças (\emph{Licenses}) \\
    \hline
    ENV & Ambiente Técnico (\emph{Technical Environment}) -
    Ferramentas, Sistema Operacional, Linguagem de Programação,
    Ambiente de Desenvolvimento. \\
    \hline
    DFCT & Número de \emph{Commits} e Relatórios de Defeitos
    (\textit{Number of Commits and Bug Reports}) \\
    \hline
    MST & Facilidade de Manutenção e Estabilidade
    (\emph{Maintainability and Stability}) \\
    \hline
    CM & Gestão de Configuração
    (\emph{Configuration Management}) \\
    \hline
    PP1 & Planejamento de Projeto Parte 1
    (\emph{Project Planning Part 1}) \\
    \hline
    REQM & Gestão de Requisitos
    (\emph{Requirements Management}) \\
    \hline
    RDMP1 & Disponibilidade de um Plano
    (\emph{Availability of a Roadmap}) \\
    \hline
  \end{tabular}
  \caption{Elementos essenciais no nível básico do OMM}
  \label{tab:omm-basic}
\end{table}

\begin{table}
  \centering
  \begin{tabular}{|p{2cm}|p{13cm}|}
    \hline
    RDMP2 & Desenvolvimento de um Plano
    (\textit{Implementation of a Roadmap}) \\
    \hline
    STK & Relações entre Interessados
    (\textit{Relationship between Stakeholders}) - Usuários,
    Desenvolvedores etc. \\
    \hline
    PP2 & Planejamento de Projeto Parte 2
    (\textit{Project Planning Part 2}) \\
    \hline
    PMC & Monitoramento e Controle do Projeto
    (\textit{Project Monitoring and Control}) \\
    \hline
    TST1 & Testes Parte 1
    (\textit{Test Part 1}) \\
    \hline
    DSN1 & Projeto Parte 1
    (\textit{Design Part 1}) \\
    \hline
    PPQA & Garantia de Qualidade no Processo e no Projeto
    (\textit{Process and Project Quality Assurance}) \\
    \hline
  \end{tabular}
  \caption{Elementos essenciais no nível intermediário do OMM}
  \label{tab:omm-intermediate}
\end{table}

\begin{table}
  \centering
  \begin{tabular}{|p{2cm}|p{13cm}|}
    \hline
    PI & Integração do Produto
    (\textit{Product Integration}) \\
    \hline
    RSKM & Gestão de Risco
    (\textit{Risk Management}) \\
    \hline
    TST2 & Testes Parte 2
    (\textit{Tests Part 2}) \\
    \hline
    DSN2 & Projeto Parte 2
    (\textit{Design Part 2}) \\
    \hline
    RASM & Resultados das Avaliações de Terceiros
    (\textit{Results of $3^{rd}$ Party Assessments}) \\
    \hline
    REP & Reputação
    (\textit{Reputation}) \\
    \hline
    CONT & Contribuições
    (\textit{Contributions}) \\
    \hline
  \end{tabular}
  \caption{Elementos essenciais no nível avançado do OMM}
  \label{tab:omm-advanced}
\end{table}

O texto do OMM apresenta uma abordagem Objetivo Pergunta Métrica (GQM
- \textit{Goal Question Metric}) no qual cada elemento possui um
conjunto de objetivos que precisam ser alcançados (ou não, dependendo
do tipo de organização sendo avaliada). As perguntas são mapeadas para
práticas recomendadas com detalhes de itens que deveriam ser
encontrados para validar que a prática é seguida.

O resto do documento de descrição do OMM apresenta recomendações para
os diferentes tipos de entidades que poderiam se interessar em obter
uma certificação OMM. Também existe uma descrição extensa de como deve
ser realizada a avaliação de uma entidade com um questionário e
informações sobre como cada prática pode ser avaliada.

A próxima Seção apresenta como a Programação Extrema descrita por Kent
Beck pode ser mapeada para os elementos de confiabilidade necessários
em cada nível do OMM.

\section{Um mapeamento de Programação Extrema para o OMM}
\label{sec:xp-em-omm}

Num primeiro momento, é importante explicar porque foi escolhida a
Programação Extrema ao invés de métodos ágeis em geral. Apesar dos
Métodos Ágeis apontarem valores e princípios, não chegam a descrever
práticas ou atividades nem fluxos de trabalho. Desta forma, é
necessário optar por algum método ágil específico que faça essa
descrição mais detalhada.

Programação Extrema é um dos métodos ágeis mais antigos e mais
estudados. Graças a isso, suas várias práticas e fluxos já foram
bastante analisados o que permite mapear diversas práticas a
resultados desejados.

A próxima subseção (Seção~\ref{sec:xp+omm}) apresenta práticas da
Programação Extrema que contribuem para atingir algum objetivo do
nível básico do OMM. A Seção~\ref{sec:xp-omm-intermediate} apresenta
os elementos essenciais dos níveis intermediários e avançados do OMM e
as práticas de XP que contribuem para atingi-los.  Em seguida, a
Seção~\ref{sec:omm-resumo} apresenta um resumo dos usos das práticas
de XP para atingir os objetivos essenciais do OMM e quais deles não
são abordados. Por fim, a Seção~\ref{sec:xp-omm} apresenta uma
avaliação do papel que o OMM pode ter nas comunidades de software
livre e alguns desafios que serão encontrados.

\subsection{Práticas de Programação Extrema que contribuem com o OMM
  básico}
\label{sec:xp+omm}

O OMM no nível básico se divide em 11 elementos essenciais. Essa Seção
está subdividida de acordo com os elementos cobertos por alguma
prática de Programação Extrema. Sendo assim, as próximas subseções
abordarão cada uma dos elementos essenciais e as práticas que ajudam a
atingir seu objetivo.

\subsubsection{Documentação do Produto (PDOC)}
\label{sec:+pdoc}

Uma das críticas comuns à Programação Extrema é que não se cria
nenhuma documentação sobre o sistema. Para abordar esse assunto, é
importante notar que existem dois tipos diferentes de documentação
conforme descrito no Objetivo PDOC 1 do OMM. A documentação para
desenvolvedores e a documentação para usuários. O Objetivo PDOC 3
ainda cita uma terceira forma de documentação que é a documentação
geral do projeto e consiste em aspectos comuns das documentações
anteriores além de documentação sobre as próprias documentações.

Com relação à documentação para desenvolvedores, a prática de
Desenvolvimento Dirigido por Testes (TDD - \textit{Test Driven
  Development}) e sua complementação Desenvolvimento Dirigido por
Comportamento (BDD - \textit{Behavior Driven Development}) são
práticas de \textit{design} mas possuem o efeito colateral de prover
diversos exemplos de uso do código criado. A ideia do BDD separa o
teste no sentido de verificação de entrada e saída do comportamento
desejado com aquele teste graças ao uso consciente de nomes de testes
que descrevam o comportamento desejado.

Diversas ferramentas de BDD (como
JBehave\footnote{\url{http://jbehave.org/} - Último acesso
  29/09/2010}, RSpec\footnote{\url{http://rspec.info/} - Último acesso
  29/09/2010} e
DocTest\footnote{\url{http://docs.python.org/library/doctest.html} -
  Último acesso 29/09/2010}) possuem relatórios de execução que
produzem saídas em formato de documentos. Esses relatórios apresentam
descrições de como os módulos do sistema funcionam de acordo com os
testes que foram escritos para eles. Dessa forma, não somente o
sistema ganha uma documentação extensa, como também existe a garantia
que esta documentação será mantida, atualizada ou, pelo menos,
informará que o sistema mudou com relação a ela.

Dessa forma, a prática de TDD/BDD ajuda a atingir o objetivo descrito
pela Prática PDOC 1.1 que exige a criação de uma documentação para
desenvolvedores. Caso o relatório de documentação seja gerado
automaticamente na construção do projeto, essa prática também ajuda a
atingir o objetivo PDOC 3 que pede que a documentação seja melhorada
com o produto.

Ainda com relação à área de Documentação (PDOC), a prática PDOC 1.3,
que pede pela criação de documentações genéricas do produto, pode ser
cumprida com a realização do planejamento da iteração e a coleta de
seus resultados em uma ferramenta online (como
XPlanner\footnote{\url{http://xplanner.org/} - Último acesso em
  29/09/2010},
Mingle\footnote{\url{http://studios.thoughtworks.com/mingle-agile-project-management}
  - Último acesso em 29/09/2010},
Calopsita\footnote{\url{http://calopsitaproject.com/} - Último acesso
  em 29/09/2010} etc). Dessa forma, a documentação do que foi
planejado em cada fase do produto assim como o andamento até o momento
naquela fase é atualizada automaticamente conforme os desenvolvedores
completam suas tarefas.

\subsubsection{Uso de Padrões Estabelecidos e Adotados (STD)}
\label{sec:+std}

Apesar de Kent Beck não mencionar nenhuma prática com relação às
ferramentas usadas no desenvolvimento dos projetos, as práticas de
Código Compartilhado e de \textit{Design} Simples encorajam o
desenvolvimento de aplicações para as quais é fácil um desenvolvedor
participar ativamente do desenvolvimento de um projeto em pouco
tempo. Sendo assim, o uso de padrões abertos amplamente difundidos e
utilizados reduz a necessidade de treinamento.  Portanto, pode-se
argumentar que a adesão a padrões abertos no produto é uma prática que
apoia as práticas de Código Compartilhado e \textit{Design} Simples e,
ao mesmo tempo, é apoiada por elas.

% XXX Citação do Agile State Survey da VersionOne

Além disso, a adoção de Programação Extrema é compatível com o
Objetivo STD 2 - Adotar processos de desenvolvimento padrões - já que
Programação Extrema é um processo aberto e livre. O relatório do
estado de métodos ágeis de 2010 realizado pela VersionOne indica que
21\% das equipes que usam métodos ágeis utilizam Programação Extrema
pura ou aliada a outro método (como Scrum).

\subsubsection{Qualidade do Plano de Testes (QTP)}
\label{sec:+qtp}

Novamente, a prática de TDD/BDD e suas variações como Desenvolvimento
Dirigido por Testes de Aceitação (ATDD - \textit{Acceptance Test
  Driven Development})~\cite{Owen2004}, apesar de serem técnicas
essencialmente ligadas ao \textit{design} do projeto, tem como efeito
colateral a criação e manutenção de testes em vários níveis (Testes de
unidade, de integração e de sistema ou integração de sistema
dependendo do objetivo do projeto). Dessa forma, o uso conjunto de
TDD, BDD e ATDD permite atingir o Objetivo QTP 1 (Prover um plano de
alta qualidade de testes) já que o plano é a descrição e implementação
dos testes logo antes da realização da funcionalidade e decorre
diretamente do plano de desenvolvimento.

Também cobre-se o Objetivo QTP 2 (Implementar e gerir o processo de
testes) com essas práticas de XP já que o desenvolvimento e execução
dos testes é garantida antes, durante e após a implementação das
funcionalidades associadas. A prática de Integração Contínua contribui
para o Objetivo QTP 2 ao garantir que os testes são realizados
frequentemente a cada modificação da base de código garantindo
\textit{Feedback} imediato do trabalho necessário.

TDD/BDD e ATDD também garantem que o Objetivo QTP 3 (Melhorar o
processo de testes) será atingido já que forçam o desenvolvedor a
incluir um teste antes de realizar qualquer mudança no código do
sistema. Dessa forma, correções de erros, inclusões de funcionalidades
ou melhorias de desempenho são capturadas em testes automatizados e
garantem a adequação do futuro do projeto a essas decisões.

\subsubsection{Ambiente Técnico (ENV)}
\label{sec:+env}

As práticas de Código Compartilhado, Repositório Único de Código e
Integração Contínua exigem uma organização do projeto de forma a
facilitar ao máximo a reprodução do ambiente de desenvolvimento e de
produção. Desta forma, o repositório único de código deveria conter
todos os arquivos necessários para construir e rodar os testes
automatizados do sistema. Sendo assim, tratam-se vários pontos
apontados pelo Objetivo ENV 1 (Planejar o desenvolvimento de recursos
e infra-estrutura) já que basta saber qual o repositório único de
código para conseguir montar um ambiente de contribuição ao projeto.

A prática de Código Compartilhado exige uma padronização no estilo de
escrita de código e de teste que deve ser compartilhado por
todos. Sendo assim, os arquivos descrevendo essas padronizações devem
estar sob controle de versão no repositório único de código de forma a
serem obtidos por qualquer contribuidor.

Por fim, a prática de Integração Contínua exige um sistema automático
para obtenção do código, instalação do mesmo num ambiente limpo e
execução dos testes automatizados do projeto. Porém, para que seja
possível montar esse tipo de ambiente automaticamente, é necessário
que a descrição e configuração do ambiente de produção esteja incluída
nessa ferramenta de Integração Contínua servindo tanto de exemplo
quanto de documentação.

\subsubsection{Número de \textit{Commits} e Relatório de Defeitos
  (DFCT)}
\label{sec:+dfct}

A prática de Repositório Único de Código exige o uso de uma ferramenta
de controle de versão. Essa ferramenta permite manter o histórico de
qualquer alteração realizada nos arquivos sob controle de
versão. Dessa forma, é muito fácil obter qual o número de
\textit{commits} realizados e o que cada \textit{commit} procurava
resolver.

Com relação ao relatório de defeitos, a prática de Histórias exige que
qualquer mudança que precisa ser realizada no projeto passe pela
escrita de uma História. Sendo assim, qualquer relatório de defeito
pode ser apresentado na forma de uma tarefa que precisa ser realizada
e descrita como uma História. Dessa forma, relatar um defeito é
equivalente a escrever uma História e inserí-la no conjunto de
histórias do projeto.

O uso de ferramentas para gestão de projeto \emph{online} (como as
citadas na Seção~\ref{sec:+pdoc}) permite que o cadastro dessas
Histórias seja realizado de forma simples e direta. A forma de
contribuir com alterações de código ou documentação relacionados a uma
determinada História varia um pouco de acordo com o sistema de
controle de versões usado. Em ferramentas para controle de versão
distribuídas, a melhor forma de contribuir com sugestões de mudanças é
apenas incluir um \textit{link} para o controle de versão com as
mudanças. Numa ferramenta de controle de versão tradicional, o melhor
seria gerar um arquivo com as diferenças entre o código original e o
código alterado. De qualquer forma, incluir essas sugestões é tão
simples quanto anexar um arquivo ou incluir um \textit{link}.

Dessa forma, cobre-se o Objetivo DFCT 1 ao prover uma forma
padronizada e simples de contribuir com o projeto. O uso de uma
ferramenta de controle de versão atualizada (como pede a prática
Repositório Único de Código) também ajuda a atingir o Objetivo DFCT 2
que exige uma gestão das contribuições, \textit{commits} e relatórios
de erros.

\subsubsection{Facilidade de Manutenção e Estabilidade (MST)}
\label{sec:+mst}

As práticas de Refatoração e Programação em Pares são essenciais em XP
para reduzir o número de defeitos inseridos no código mas também para
garantir que o código da aplicação permaneça legível e mais simples de
manter. Um dos estudos mais reconhecidos sobre Programação em Pares
\cite{Williams2000} mostra que o uso de Programação em Pares reduz a
quantidade de defeitos, melhora a qualidade do \textit{design} e
distribui conhecimento técnico. Já a Refatoração~\cite{Refac01} é uma
prática cujo objetivo é melhorar a manutenabilidade de um determinado
código sem alterar sua funcionalidade.

Essas duas práticas aliadas ajudam a cumprir o Objetivo MST 1 que
requer um planejamento para qualidade do produto em termos de
requisitos não-funcionais. Adicionando a prática de Código e Teste é
possível garantir também que o sistema se comporte como esperado em
diversos ambientes (\textit{hardware} e software).

Já a prática de Retrospectiva aliada com a prática de Análise de Causa
Inicial irão permitir atingir o Objetivo MST 2 que procura melhorar a
qualidade do processo do projeto. A ideia da prática de Retrospectiva
\cite{Derby2006} é usar técnicas de identificação de problemas ou
pontos de melhoria e coleta de sugestões de mudanças para realizar
essas melhoras. Já a Análise de Causa Inicial permite resolver não
apenas os problemas superficiais mas também os problemas que são
fontes de problemas superficiais.

Por fim, a prática de Integração Contínua aliada com a de TDD/BDD vai
permitir atingir o Objetivo MST 3 que pede para gerenciar o processo
de manutenabilidade. Essa gestão se dará com os resultados da
construção e execução dos testes realizados a cada mudança no
Repositório Único de Código. Tendo esses relatórios de resultado da
construção, é muito fácil descobrir qual foi o \textit{commit} exato
que retirou uma determinada funcionalidade ou interface de programação
já que cada construção e relatório de testes estão atrelados a um
\textit{commit}.

\subsubsection{Gestão de Configuração (CM)}
\label{sec:+cm}

A prática de Código Compartilhado pede que qualquer desenvolvedor do
sistema tenha o direito e a possibilidade de alterar qualquer trecho
de código independente de quem foi o autor desse trecho ou quando ele
foi criado. Para que isso seja possível, o projeto deve estabelecer
padrões de estrutura de arquivos, formatação usada além de estilo de
nomenclatura e outros itens que permitam uma integridade de forma e de
estilo para o código do projeto.

Esses padrões iniciais estabelecem uma configuração padrão para
qualquer desenvolvedor e podem ser integrados ao sistema de controle
de versão do Repositório Único de Código para garantir a gestão e
distribuição desses arquivos de forma consistente. Esses arquivos
contribuem com o Objetivo CM 1 que pede o estabelecimento de linhas de
base para o desenvolvimento do produto.

As práticas de História e de Ciclos (Semanais e de Estação) permitem
também manter um acompanhamento de quando as Histórias são criadas e
quando elas são planejadas para serem desenvolvidas. Manter o
histórico dessas Histórias e Ciclos permite atingir o Objetivo CM 2 de
acompanhar e controlar mudanças nas configurações do projeto.

\subsubsection{Planejamento de Projeto Parte 1 (PP1)}
\label{sec:+pp1}

A prática de Jogo do Planejamento pede que a equipe de desenvolvimento
trabalhe com o cliente ou usuário do sistema e decida qual será o
trabalho a ser realizado no próximo Ciclo Semanal. Durante essa
reunião, o usuário define quais são as Histórias que precisam ser
realizadas com maior prioridade. Seguindo essa ordem de prioridades,
os desenvolvedores pensam sobre a dificuldade para realizar aquela
História considerando suas experiências passadas e estimando
comparativamente. Dessa forma, estabelecem-se estimativas para as
Histórias mais importantes até que o cliente consiga escolher um
conjunto de histórias que preencha o Ciclo Semanal.

Dessa forma, ao final de um Jogo do Planejamento, o projeto tem uma
estimativa para o escopo a ser realizado durante aquele ciclo, o
esforço/custo e duração para realização dessas tarefas e uma
estimativa crua de tarefas futuras. Essas estimativas ajudam a atingir
o Objetivo PP1 1 que pede o estabelecimento de estimativas.

O Jogo do Planejamento ainda diz que as Histórias que forem
consideradas de maior valor de negócio e de maior risco são as que tem
maior prioridade no desenvolvimento. Dessa forma, a equipe precisa
informar junto com a estimativa da história qual o risco técnico
associado com aquela história e o cliente deve saber o risco de
negócios relacionado. Com essas duas informações, o plano de
desenvolvimento do projeto decorre da regra de sempre trabalhar nas
histórias de maior risco e maior valor existente atualmente. Dessa
forma, caso o projeto encontre um grande problema, a falha acontecerá
cedo no processo reduzindo o custo necessário para identificar um
problema sério.

Esse planejamento de desenvolvimento do projeto é o Objetivo PP1 2 que
corresponde ao Ciclo de Estação do Produto.

\subsubsection{Gestão de Requisitos (REQM)}
\label{sec:+reqm}

A prática de Histórias consiste em coletar de forma concisa e curta
uma necessidade do ponto de vista de negócio para o projeto. Essa
história é inicialmente apresentada como uma descrição muito curta e
muito simples mas é associada com a pessoa que a requisitou. Dessa
forma, caso essa História seja escolhida para o Ciclo Semanal, os
desenvolvedores tem a possibilidade de obter mais informações sobre o
trabalho a ser realizado. As reuniões do Jogo do Planejamento permitem
revisar e atualizar essas histórias de forma a garantir a gestão com
relação às mudanças nos requisitos descritos pela história. Com isso,
é possível cumprir o Objetivo REQM 1 que pede uma gestão de
requisitos.

\subsection{Práticas de XP que contribuem para o OMM nível
  Intermediário e Avançado}
\label{sec:xp-omm-intermediate}

Conforme se avançam nos níveis do OMM, assim como para o CMM, as
exigências se tornam mais rígidas e difíceis de cumprir. O nível
Intermediário do OMM trabalha com 7 elementos essenciais (Tabela
\ref{tab:omm-intermediate}) enquanto o nível avançado envolve mais 7
elementos essenciais (Tabela~\ref{tab:omm-advanced}).

No nível intermediário, adicionam-se os elementos:
\begin{itemize}
\item \textit{RDMP2} - Desenvolvimento de um Plano
\item \textit{PP2} - Planejamento de Projeto Parte 2
\item \textit{STK} - Relações entre Interessados (\textit{STK})
\item \textit{PMC} - Monitoramento e Controle do Projeto
\item \textit{PPQA} - Garantia de Qualidade no Processo e no Projeto
\item \textit{TST1} - Testes Parte 1
\item \textit{DSN1} - Projeto Parte 1
\end{itemize}

No nível avançado, completa-se com:
\begin{itemize}
\item \textit{TST2} - Testes Parte 2
\item \textit{DSN2} - Projeto Part 2
\item \textit{PI} - Integração do Produto
\item \textit{RSKM} - Gestão de Risco
\item \textit{RASM} - Resultado das Avaliações de Terceiros
\item \textit{REP} - Reputação
\item \textit{CONT} - Contribuições
\end{itemize}

Em ambos níveis, práticas de XP abordam diretamente os assuntos
tratados por alguns desses elementos essenciais. No que diz respeito à
extensão do Plano do Projeto, a principal exigência é de que sejam
planejadas atualizações ao plano do projeto e que elas sejam
seguidas. Em XP, a prática de Ciclo Semanal pede que a equipe
renegocie o seu trabalho da semana. Essa negociação deve impactar como
for necessário o planejamento do projeto. Dessa forma, é possível e
até provável que se realize um Jogo do Planejamento curto no início da
cada Ciclo Semanal. Sendo assim, pode-se argumentar que as
atualizações ao plano do projeto são realizadas no início de cada
ciclo i.e. periodicamente como pede o elemento \textit{RDMP2}.

No que diz respeito ao elemento \textit{PP2}, a principal adição é com
relação a obter comprometimento com o plano traçado. Nesse aspecto,
não há, nominalmente, nenhuma prática de XP que aborde esse
assunto. No entanto, existe um princípio em métodos ágeis e na forma
com que se conduz um Jogo do Planejamento que busca este
objetivo. Esse princípio exige que a equipe de desenvolvimento seja
responsável por estimar as tarefas que precisam ser realizadas e
determine quais consegue cumprir no tempo disponível. A ideia de que o
próprio time estabeleça as estimativas e o plano cria um
comprometimento natural já que as pessoas são tomadoras de decisões e
não meros acatadores. Esse princípio é chamado de Responsabilidade
Aceita.

A parte de Testes exigida pelo nível intermediário menciona uma
preparação do projeto para verificação, uma revisão externa e uma
verificação dos produtos livres usados pelo projeto. A prática de
Programação em Pares tem um feito óbvio de revisão externa constante e
imediata. Dessa forma, pode-se argumentar que a validação externa é
realizada constantemente pelo pessoa que está fazendo par com o
programador. Além disso, pode-se dizer também que as práticas de TDD e
ATDD colaboram para facilitar a criação de um ambiente de verificação
já que essas práticas forçam os programadores a criar arquiteturas que
facilitem a elaboração de testes automatizados.

Com relação ao elemento de Monitoração e Controle do Projeto, o
principal objetivo é monitorar se o projeto está seguindo o plano e,
se não estiver, definir medidas corretivas e implementá-las. Nesse
contexto, as práticas de Retrospectiva e Análise de Causa Raiz
permitem que a equipe identifique, analise e entenda desvio no plano
do projeto, seus motivos e quais podem ser ações corretivas que
resolvam não apenas o problema imediato mas também as chances desses
problema se repetir no futuro.

Os aspectos de Projeto Parte 1, Garantia de Qualidade no Processo e no
Projeto e Relações entre Interessados não são diretamente tratados por
nenhuma prática de XP. No caso da Garantia de Qualidade, a prática de
Retrospectiva tem um efeito indireto mas depende muito da capacidade
do time em encontrar melhorias para o processo e projeto. Na parte de
Projeto, as exigências são bastante específicas com relação à obtenção
e rastreabilidade das funcionalidades. A prática em XP que cuida
desses aspectos são as práticas de Histórias e de Sentar Junto. No
entanto, a rastreabilidade e qualidade da obtenção da história
dependem bastante da habilidade das pessoas envolvidas no projeto.

Pro nível avançado, os objetivos são mais exigentes e, apesar do uso
de diversas práticas de XP em conjunto conseguirem atingir alguns
deles, os resultados ainda são muito dependentes do habilidade e
experiência dos envolvidos para continuarem perseguindo os
objetivos. Desta forma, para poder garantir que o processo atinja os
objetivos desejados, será necessário incluir algumas práticas àquelas
apresentadas anteriormente.

\subsection{Resumo}
\label{sec:resumo-omm}

Em resumo, a Tabela~\ref{tab:omm-basic-by-xp} apresenta um mapeamento
entre práticas de Programação Extrema e elementos essenciais do OMM no
nível básico. Graças à tabela, percebe-se que o elemento de
documentação é o ponto menos abordado pelas práticas de Programação
Extrema enquanto o elemento de Manutenabilidade e Estabilidade é o
mais coberto.

\begin{table}
  \centering
  \begin{tabular}{|p{4cm}|c|c|c|c|c|c|c|c|c|}
    \hline
    & PDOC & STD & QTP & ENV & DFCT & MST & CM & PP1 & REQM \\
    \hline
    Código Compartilhado & & $\surd$ & & $\surd$ & & $\surd$ & $\surd$ & & \\
    \hline
    \textit{Design} Simples & & $\surd$ & & & & & & & \\
    \hline
    Repositório Único de Código & & & & $\surd$ & $\surd$ & $\surd$ & $\surd$ & & \\
    \hline
    Integração Contínua & & & $\surd$ & $\surd$ & & & $\surd$ & & $\surd$ \\
    \hline
    Programação em Pares & & $\surd$ & & & & $\surd$ & & & \\
    \hline
    Código e Teste & & & $\surd$ & & & $\surd$ & & & \\
    \hline
    TDD & $\surd$ & & $\surd$ & & $\surd$ & $\surd$ & & & \\
    \hline
    Refatoração & & & & $\surd$ & & $\surd$ & & & \\
    \hline
    Ciclo Semanal & & & & & & & $\surd$ & $\surd$ & \\
    \hline
    Ciclo de Estação & & & & & & & $\surd$ & $\surd$ & $\surd$ \\
    \hline
    Retrospectiva & & & & & & $\surd$ & & & \\
    \hline
    Análise de Causa Inicial & & & & & & $\surd$ & & & \\
    \hline
    Histórias & & & & & $\surd$ & & $\surd$ & $\surd$ & $\surd$ \\
    \hline
    Jogo do Planejamento & & & & & & & & $\surd$ & $\surd$ \\
    \hline
  \end{tabular}
  \caption{Mapeamento de elementos essenciais do OMM nível básico com relação às práticas de XP}
  \label{tab:omm-basic-by-xp}
\end{table}

Já a Tabela~\ref{tab:omm-intermediary-by-xp} mostra como fica mais
difícil cumprir as exigências essenciais no nível intermediário. Em
especial, as marcas com asterisco mostram práticas que abordam o
assunto mas dependem da capacidade da equipe de buscar o objetivo
procurado. Isto é, não basta a equipe aplicar a prática
corretamente. Ela precisa buscar o objetivo exigido pelo elemento
essencial e a prática apenas auxilia atingir esse objetivo. Vale notar
que nenhuma prática de XP aborda (nem indiretamente) os elementos
\textit{STK} e \textit{DSN1}.

\begin{table}
  \centering
  \begin{tabular}{|p{6cm}|c|c|c|c|c|c|c|c|c|}
    \hline
    & RDMP2 & PP2 & STK & PMC & PPQA & TST1 & DSN1\\
    \hline
    Programação em Pares & & & & & & $\surd$ & \\
    \hline
    Código e Teste & & & & & & $\surd$ & \\
    \hline
    TDD & & & & & & $\surd$ & \\
    \hline
    Ciclo Semanal & $\surd$ & & & & & & \\
    \hline
    Ciclo de Estação & $\surd$ & & & & & & \\
    \hline
    Retrospectiva & & & & $\surd$ & & & \\
    \hline
    Análise de Causa Inicial & & & & $\surd$ & & & \\
    \hline
    Histórias & & & & & * & & \\
    \hline
    Sentar Junto & & & & & * & & \\
    \hline
    Jogo do Planejamento & $\surd$ & * & & & & & \\
    \hline
  \end{tabular}
  \caption{Mapeamento de elementos essenciais do OMM nível
    intermediário com relação às práticas de XP}
  \label{tab:omm-intermediary-by-xp}
\end{table}

A Seção a seguir (Seção~\ref{sec:xp-omm}) apresenta os elementos
essenciais e partes de objetivos que não foram atingidos com as
práticas de XP listadas até agora.

\subsection{Elementos essenciais não cobertos pela Programação
  Extrema}
\label{sec:xp-omm}

Os principais elementos do nível básica que não foram cobertos pelas
práticas de XP são:

\begin{enumerate}
\item PDOC - Objetivo 1.2

  - Criar uma documentação para usuários
  \label{omm:pdoc1.2}
\item PDOC - Objetivo 2

  - Criar um a documentação pro produto
  \label{omm:pdoc2}
\item STD - Objetivo 3

  - Garantir independência estratégica do projeto
  \label{omm:std3}
\item LCS

  Toda a parte de Licenças
  \label{omm:lcs}
\item ENV - Objetivo 3

  Melhorar o uso de ferramentas livres
  \label{omm:env3}
\item CM - Objetivo 3

  Estabelecer Integridade
  \label{omm:cm3}
\item RDMP1 - Objetivo 1

  Planejar o plano do produto
  \label{omm:rdmp11}
\end{enumerate}

Com relação ao Item~\ref{omm:pdoc1.2}, XP defende que a documentação
para o usuário deve ser incluída como uma História como qualquer
outra. Nesse ponto, ela pode ser priorizada e incluída no Ciclo
Semanal como qualquer outra funcionalidade. Sendo assim, o processo
não garante que a documentação para o usuário será desenvolvida mas
garante que, se ela for importante, ela será realizada.

Nesse ponto, a melhor opção pode ser incluir a escrita de documentação
para usuário junto com a história da funcionalidade. Dessa forma,
garante-se que a História não é aprovada (ou terminada) caso não
exista essa documentação.

O Item~\ref{omm:pdoc2} cai numa situação similar. Implementar uma
documentação embutida no produto é uma funcionalidade como outra que
deveria ser priorizada e incluída no Ciclo Semanal como outras. O
ponto a destacar é apenas que, uma vez que o sistema estiver
desenvolvido, ele deveria ser realizado de tal forma que a
documentação para o usuário refletisse na documentação para o produto
em si.

O Item~\ref{omm:std3} é um pouco mais complicado de tratar já que
envolve analisar cada um dos padrões para avaliar a chance dele causar
uma dependência futura. Para conseguir atingir esse objetivo, é
necessário incluir alguma prática de análise de risco para os
componentes usados na implementação de cada História.

O Item~\ref{omm:lcs} é similar mas, em geral, terá um peso mais
significativo apenas no início do projeto. Sendo assim, talvez seja
importante dedicar um Ciclo Semanal de preparação e análise do projeto
para averiguar as possibilidades no quesito de licenças que o projeto
aceita.

O Item~\ref{omm:env3} envolve pesquisa em busca de componentes abertos
para substituir componentes proprietários. Dessa forma, o processo
precisaria incluir inspeções recorrentes dos componentes e ferramentas
usadas em busca de alternativas livres.

O Item~\ref{omm:cm3} é provavelmente um dos objetivos mais difíceis de
atingir. Integridade Conceitual é um requisito que já é difícil de
atingir quando uma única pessoa realiza todo o trabalho. No contexto
de uma equipe com contribuições externas (como o cenário de software
livre), transmitir o conceito do projeto é bem complicado. Nesse caso,
a sugestão seria de incluir pequenas conversas entre os
usuários/clientes e desenvolvedores gravadas (aúdio e vídeo)
disponíveis na página Internet do projeto.

O Item~\ref{omm:rdmp1} pode ser obtido ao estabelecer uma cadência
para os Ciclos de Estação e realizar Jogos do Planejamento em vários
níveis ao final de cada estação. XP defende que esse tipo de
planejamento a longo prazo deveria ser realizado apenas num nível bem
geral para permitir mudanças que ocorrerão inevitavelmente.

Considerando essas possíveis justificativas para aderir ao OMM nível
básico, percebe-se que métodos ágeis e, em especial nesse caso,
Programação Extrema são bons candidatos para atingir uma certificação
OMM nível básico. Com isso, espera-se que comunidades de software
livre possam atingir esse objetivo sem forçar o uso de uma metodologia
tradicional aos seus contribuidores.

\section[OMM no contexto livre]{O OMM como semente para uso de
  metodologias em ambientes livres}
\label{sec:sl+omm}

\begin{quote}
  ``Toda unanimidade é burra'' -- Nelson Rodrigues
\end{quote}

O universo do software livre se inspira muito nessa
ideia. Diversidade, alternativas, mudanças de direções, opiniões e até
mesmo brigas são elementos essenciais para a evolução do eco-sistema
livre. Qualquer tentativa de buscar uma única forma de atingir um
objetivo nesse ambiente é falha de antemão. No entanto, o objetivo do
OMM não é esse. O OMM busca apenas ajudar empresas e comunidades a
avaliarem seus próprios processos e produtos sob uma ótica de
viabilidade em uso comercial.

Nesse sentido, o uso de metodologias ágeis é apenas uma das opções
existentes.  Não é a única e muito menos a melhor; assim como o
próprio objetivo do OMM não é o único nem o mais buscado pelos
projetos livres existentes. Com certeza, muitas críticas virão e
muitas outras opções serão apresentadas. O OMM já apresenta uma boa
forma de lidar com isso já que, em seu próprio plano, apresenta ideias
para receber sugestões e melhorias e incorporá-las conforme for
possível.  Se for bem sucedido nessa empreitada, o OMM ganhará o
respeito da comunidade de software livre.

Um dos possíveis problemas do OMM é a característica herdada do CMM de
dar muita importância ao planejamento e ao plano. Métodos ágeis seguem
a filosofia de que ``Planos não tem valor, mas planejar é
tudo''. Sendo assim, muitos elementos essenciais do OMM que avaliam a
aderência ao plano ferem a capacidade de reagir às mudanças
rapidamente.

No entanto, este capítulo apresentou uma possível adequação de
Programação Extrema para atingir alguns objetivos traçados pelo
OMM. Projetos de software livre como o
LaunchPad\footnote{\url{http://launchpad.net} -- Último acesso em
  07/01/2011} e o
Calopsita\footnote{\url{http://calopsita.caelum.com.br} -- Último
  acesso em 07/01/2011} que, oficialmente, são desenvolvidos com
métodos ágeis poderão tomar como base essa argumentação e, se tiverem
sucesso, incentivarão outros projetos a seguirem seus passos.
