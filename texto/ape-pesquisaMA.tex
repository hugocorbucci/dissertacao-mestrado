\chapter{Pesquisa para praticantes de Métodos Ágeis}
\label{ape:MA}

\singlespacing

A pesquisa abaixo é uma versão traduzida para o Português e adaptada
para impressão da pesquisa disponibilizada em
\url{http://www.ime.usp.br/~corbucci/agile-survey.html} do dia 1$^o$
de Outubro de 2009 até o dia 1$^o$ de Dezembro de 2009.

\begin{enumerate}
\item País de residência: \verb=_________________=

\item Ano de nascimento: \verb=______=

\item Número de projetos que usavam princípios ágeis que participou
  (escolha apenas 1):
  \begin{itemize}
  \item[( )] 0
  \item[( )] 1 a 5
  \item[( )] 6 a 10
  \item[( )] 11 a 50
  \item[( )] 51 a 100
  \item[( )] Mais de 100
  \end{itemize}

\item Ano do primeiro projeto que usava princípios ágeis que
  participou: \verb=______=

\item Qual é seu principal papel (escolha apenas 1) nos projetos ágeis
  em que participa?
  \begin{itemize}
  \item[( )] Gerente de projeto
  \item[( )] Líder de equipe
  \item[( )] Programador
  \item[( )] Analista de Qualidade
  \item[( )] Testador
  \item[( )] Acompanhador
  \item[( )] Documentador
  \item[( )] Outro: \verb=_________________=
  \end{itemize}

\item Qual número médio de integrantes (escolha apenas 1) nas equipes
  dos projetos ágeis que participou?
  \begin{itemize}
  \item[( )] 1 a 5
  \item[( )] 6 a 10
  \item[( )] 11 a 20
  \item[( )] 21 a 50
  \item[( )] 51 a 100
  \item[( )] Mais de 100
  \end{itemize}

\item Já trabalhou em projetos ágeis com uma equipe (ou parte dela)
  distribuída?
  \begin{itemize}
  \item[( )] Sim
  \item[( )] Não
  \end{itemize}

\item Qual é (ou foi) o principal canal de comunicação (escolha apenas
  1) entre essa equipe?
  \begin{itemize}
  \item[( )] Face a face
  \item[( )] Site na Internet
  \item[( )] Lista de correio eletrônico
  \item[( )] Ferramenta de rastreamento de problemas
  \item[( )] IRC (\textit{Internet Relay Chat} ou Papo Retransmitido
    pela Internet)
  \item[( )] Mensagens Instantâneas
  \item[( )] Correios eletrônicos individuais
  \item[( )] Voz sobre IP (Skype, Ekiga, iChat, etc...)
  \item[( )] Nenhum
  \item[( )] Outro: \verb=_____________=
  \end{itemize}

\item Como você avalia a qualidade de communicação da equipe?

  Péssima \verb=---------------------------------------= Ótima

\item Qual é o principal canal de comunicação (escolha apenas 1) entre
  as equipes de seus projetos ágeis e os clientes desses projetos?
  \begin{itemize}
  \item[( )] Face a face
  \item[( )] Site na Internet
  \item[( )] Lista de correio eletrônico
  \item[( )] Ferramenta de rastreamento de problemas
  \item[( )] IRC (\textit{Internet Relay Chat} ou Papo Retransmitido
    pela Internet)
  \item[( )] Mensagens Instantâneas
  \item[( )] Correios eletrônicos individuais
  \item[( )] Voz sobre IP (Skype, Ekiga, iChat, etc...)
  \item[( )] Nenhum
  \item[( )] Outro: \verb=_____________=
  \end{itemize}

\item Como você avalia a qualidade de communicação entre essa equipe e
  o cliente?

  Péssima \verb=---------------------------------------= Ótima

\item Ordene os seguintes problemas do que mais atrapalha (ou
  atrapalhava) no seu projeto ágil distribuído ao que menos atrapalha
  (ou atrapalhava)? Dê uma nota de 1 a 8 sendo que 1 é o problema mais
  importante e 8 é o menos importante.
  \begin{itemize}
  \item[( )] Descobrir o que os usuários precisavam
  \item[( )] Descobrir qual era a próxima tarefa a ser realizada
  \item[( )] Entender como o projeto funciona do ponto de vista
    técnico
  \item[( )] Descobrir o estado atual do projeto
  \item[( )] Integrar código no repositório central
  \item[( )] Manter as informações sobre o projeto atualizadas no
    principal canal de comunicação
  \item[( )] Avaliar o trabalho realizado para identificar pontos de
    melhora
  \item[( )] Sincronizar com os outros colaboradores para atingir um
    objetivo comum
  \end{itemize}

\item Ordene as seguintes ferramentas daquela que mais resolveria os
  problemas citados anteriormente para a que menos resolveria. Dê uma
  nota de 1 a 8 sendo que 1 é a ferramenta mais importante e 8 é a
  menos importante.
  \begin{itemize}
  \item[( )] Mensagem ou Correio Eletrônico automático em caso de
    falha na montagem automática do projeto
  \item[( )] Estado dinâmico da versão em desenvolvimento a partir da
    ferramenta de rastreamento de problemas
  \item[( )] Gerenciamento da ferramenta de rastreamento de problemas
    a partir dos logs de commit do repositório de código
  \item[( )] Lançamento de nova versão a partir de um click no site do
    projeto
  \item[( )] Geração e atualização automática de gráficos de métricas
    a partir do repositório de código
  \item[( )] Ordenação colaborativa da prioridade dos problemas a
    serem endereçados pela equipe de desenvolvimento
  \item[( )] Linha do tempo no site do projeto ligada ao repositório
    de forma a facilitar análises em retrospectivas
  \item[( )] Um robô nos canais de communicação da equipe de forma a
    facilitar a comunicação assíncrona
  \end{itemize}

\item Você já contribuiu com projetos de software livre?
  \begin{itemize}
  \item[( )] Sim
  \item[( )] Não
  \end{itemize}

\item Quão ágil você avaliaria o principal projeto de software livre
  com o qual você contribui (ou contribuiu)?

  Nada ágil \verb=---------------------------------------= Muito ágil

\item Ordene os seguintes problemas do que mais atrapalha (ou
  atrapalhava) no seu projeto de software livre ao que menos atrapalha
  (ou atrapalhava)? Dê uma nota de 1 a 8 sendo que 1 é o problema mais
  importante e 8 é o menos importante.
  \begin{itemize}
  \item[( )] Descobrir o que os usuários precisavam
  \item[( )] Descobrir qual era a próxima tarefa a ser realizada
  \item[( )] Entender como o projeto funciona do ponto de vista
    técnico
  \item[( )] Descobrir o estado atual do projeto
  \item[( )] Integrar código no repositório central
  \item[( )] Manter as informações sobre o projeto atualizadas no
    principal canal de comunicação
  \item[( )] Avaliar o trabalho realizado para identificar pontos de
    melhora
  \item[( )] Sincronizar com os outros colaboradores para atingir um
    objetivo comum
  \end{itemize}

\item Ordene as seguintes ferramentas daquela que mais resolveria os
  problemas citados anteriormente para a que menos resolveria. Dê uma
  nota de 1 a 8 sendo que 1 é a ferramenta mais importante e 8 é a
  menos importante.
  \begin{itemize}
  \item[( )] Mensagem ou Correio Eletrônico automático em caso de
    falha na montagem automática do projeto
  \item[( )] Estado dinâmico da versão em desenvolvimento a partir da
    ferramenta de rastreamento de problemas
  \item[( )] Gerenciamento da ferramenta de rastreamento de problemas
    a partir dos logs de commit do repositório de código
  \item[( )] Lançamento de nova versão a partir de um click no site do
    projeto
  \item[( )] Geração e atualização automática de gráficos de métricas
    a partir do repositório de código
  \item[( )] Ordenação colaborativa da prioridade dos problemas a
    serem endereçados pela equipe de desenvolvimento
  \item[( )] Linha do tempo no site do projeto ligada ao repositório
    de forma a facilitar análises em retrospectivas
  \item[( )] Um robô nos canais de communicação da equipe de forma a
    facilitar a comunicação assíncrona
  \end{itemize}
\end{enumerate}
