\documentclass[12pt]{article}

\usepackage[brazil]{babel}
\usepackage[utf8]{inputenc}
\usepackage{indentfirst}
\usepackage{lscape}
\usepackage{latexsym}
\usepackage{amssymb}
\usepackage{textcomp}
\usepackage{graphicx,url}
\usepackage[T1]{fontenc}
\usepackage{hyperref}

\title{Resumo da dissertação:\\
  ``Métodos ágeis e software livre:\\Propostas de como aumentar a
  interação\\ entre estas duas comunidades''}

\author{Hugo Corbucci e Alfredo Goldman}

\begin{document}
 
\maketitle

Nos últimos anos, dois movimentos ganharam muita força no contexto de
desenvolvimento de software. De um lado, os chamados métodos ágeis
deixaram de ser técnicas alternativas para se tornarem a primeira
opção em termos de metodologia de desenvolvimento em muitas
empresas. Do outro, programas com licenças livres ou abertas voltaram
a competir diretamente com programas proprietários na preferência de
usuários de todos os níveis. Os contextos de sucesso de cada um desses
movimentos indicam, a princípio, uma enorme diferença entre os
problemas atacados e as soluções encontradas pelas comunidades. De
fato, um estudo realizado num evento de métodos ágeis brasileiro
mostrou que essa comunidade não tem uma participação muito forte em
comunidades de software livre.

No entanto, uma análise mais cuidadosa das soluções apresentadas pelas
duas comunidades leva a crer que a expansão desses movimentos os
aproxima mais do que os contextos indicavam. Essa aproximação aliada
ao pequeno envolvimento de desenvolvedores de métodos ágeis nas
comunidades de software livre apontam para a existência de alguns
empecilhos na mescla dos membros dessas comunidades. Este trabalho
procura identificar os empecilhos e propor formas de contorná-los ou
eliminá-los.

Para isso, foi realizado um estudo das soluções encontradas em
projetos de software livre sob o prisma dos métodos ágeis. Este estudo
mostra que muitos dos princípios defendidos pelo Manifesto
Ágil\footnote{\href{http://agilemanifesto.org/}{http://agilemanifesto.org/}}
são guias para as técnicas de desenvolvimento encontradas de forma
comum em projetos de software livre de sucesso. O papel de {\it
  commiter}, por exemplo, permite realizar uma revisão de todo código
que é integrado ao repositório principal do projeto de forma não
intrusiva no desenvolvimento realizado pelos contribuídores. Com isso,
projetos de software livre mostram uma atenção contínua à qualidade do
código e da arquitetura do projeto com uma visão geral dos
acontecimentos.

No que diz respeito aos métodos ágeis de desenvolvimento de software,
com a crescente adoção de seus valores e princípios, problemas em
contextos que fogem do ideal descrito envolvem equipes distribuídas,
desenvolvimento sem a presença constante do cliente e desenvolvimento
de aplicações cujas falhas podem ter efeitos mais graves. Nesses
contextos, algumas práticas originalmente propostas precisam de
adaptações. A prática da programação em par, por exemplo, é uma cuja
ausência pode ser minimizada pela adoção do sistema de {\it commiter}
para realizar uma revisão do código de forma a manter a qualidade e
visão geral do sistema.

Existem diversas outras práticas para as quais análises semelhantes
foram realizadas de forma a identificar possíveis adaptações de acordo
com o contexto do projeto. Essas análises mostraram que, em alguns
casos, falta um apoio ferramental que permita ou facilite a adaptação
de uma determinada prática num contexto distribuído. Para identificar
as ferramentas mais desejadas em cada uma das comunidades foram
elaborados dois questionários. Cada um dos questionários foi
direcionado a uma das comunidades e procurou identificar os principais
problemas sentidos pelos membros daquela comunidade.

Com os resultados desses questionários, foi possível identificar os
principais problemas enfrentados pelos participantes em cada um dos
contextos que lhe é familiar. Esses problemas aliados com as
ferramentas sugeridas para resolvê-los levam a uma adaptação de
métodos ágeis voltada para um contexto de software livre. Este
trabalho termina apresentando essa adaptação sob o prisma do modelo de
maturidade para desenvolvimento open source sugerido pelo projeto
QualiPSo\footnote{\href{http://www.qualipso.org}{http://www.qualipso.org}}
como uma possível implementação que se adeque às exigências
traçadas. Na tentativa de encaixar a sugestão no modelo surgiram
algumas dificuldades que apontam possíveis problemas ou pontos de
melhoria no trabalho realizado.

\end{document}
